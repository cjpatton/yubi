%
\def\camready{0}
\ifnum\camready=0
  \makeatletter
  \let\@twosidetrue\@twosidefalse
  \let\@mparswitchtrue\@mparswitchfalse
  \makeatother
\fi

\documentclass{build/llncs}
\usepackage{color,soul}
\usepackage[dvipsnames]{xcolor}
\usepackage{enumitem}
\usepackage{times}
% header.tex
%
% Formatting and common macros for crypto papers. Include this first.
\usepackage{graphics}
\usepackage[toc,page]{appendix}
\usepackage[font={small}]{caption}
\usepackage{hyperref}
\usepackage{xspace}
\usepackage{amsmath}
\usepackage{amssymb}
\usepackage{amsfonts}
\usepackage{parskip}
\usepackage{framed}
\usepackage{times}

\hypersetup{
    colorlinks,%
    citecolor=black,%
    filecolor=black,%
    linkcolor=black,%
    urlcolor=black
}

\def\dashuline{\bgroup
  \ifdim\ULdepth=\maxdimen  % Set depth based on font, if not set already
	  \settodepth\ULdepth{(j}\advance\ULdepth.4pt\fi
  \markoverwith{\kern.15em
	\vtop{\kern\ULdepth \hrule width .3em}%
	\kern.15em}\ULon}

\newcounter{foot}
\setcounter{foot}{1}
\setlength\parindent{2em}

% Editorial
\renewcommand{\paragraph}[1]{\noindent\textbf{#1}}
\newcommand{\heading}[1]{\paragraph{#1}}
\newcommand{\ala}{\textit{a la}\xspace}
\newcommand{\etal}{et al.\xspace}
\newcommand{\apriori}{\textit{a priori}\xspace}
\newcommand{\viceversa}{\textit{vice versa}\xspace}

% Fonts for various types
\newcommand{\notionfont}[1]{\textnormal{#1}\xspace}
\newcommand{\varfont}[1]{\textit{#1}}
\newcommand{\flagfont}[1]{\mathsf{#1}}
\newcommand{\vectorfont}[1]{\vec{#1}}
\newcommand{\oraclefont}[1]{\cryptofont{#1}}
\newcommand{\schemefont}[1]{\textsc{#1}\xspace}
\newcommand{\prinfont}[1]{\textsf{#1}\xspace}
\newcommand{\procfont}[1]{\mathsf{#1}}
\newcommand{\algorithmfont}[1]{\mathcal{#1}}
\newcommand{\adversaryfont}[1]{\mathit{#1}}
\newcommand{\setfont}[1]{\mathcal{#1}}
\newcommand{\cryptofont}[1]{\mathbf{#1}\hspace{0.5pt}}
\newcommand{\capgreekfont}[1]{\mathrm{#1}}

% Crypto functions
\newcommand{\Exp}[1]{\cryptofont{Exp}^{{\tiny \MakeLowercase{#1}}}}
\newcommand{\Adv}[1]{\cryptofont{Adv}^{{\tiny \MakeLowercase{#1}}}}

% Math
\DeclareMathAlphabet\mathbfcal{OMS}{cmsy}{b}{n}
\newcommand{\dqed}{\hfill$\Diamond$}
% FIXME What's the deal with this command and nested parans? This and also
% substr
\def\ceil(#1){\lceil #1 \rceil}
\def\floor(#1){\lfloor #1 \rfloor}
\newcommand{\goesto}{{\rightarrow}}

% - Sets
\newcommand{\setify}[1]{\procfont{set}\left(#1\right)}
\newcommand{\setlen}[1]{|#1|}
\newcommand{\multisetlen}[1]{\|#1\|}
\newcommand{\Z}{\mathbb{Z}}
\newcommand{\N}{\mathbb{N}}
\newcommand{\R}{\mathbb{R}}
\newcommand{\bits}{\{0,1\}}
\newcommand*\bigunion{\bigcup}
\newcommand*\bigintersection{\bigcap}
\newcommand*\union{\cup}
\newcommand{\multiunion}{\uplus}
\newcommand*\intersection{\cap}
\newcommand*\cross{\times}
\newcommand*\by{\cross}
\newcommand{\getsr}{\mathrel{\leftarrow\mkern-14mu\leftarrow}}
%\newcommand{\getsr}{\xleftarrow{\text{\tiny{\$}}}}
%\newcommand{\getsr}{{\:{\leftarrow{\hspace*{-3pt}\raisebox{.75pt}{$\scriptscriptstyle\$$}}}\:}}
\newcommand{\setop}[1]{\mathsf{set}(#1)} %^ \procfont
\def\str(#1){\procfont{set}\left(#1\right)}
\def\bydef{\stackrel{\rm def}{=}}

% - String operations
\newcommand{\emptystr}{\varepsilon}
\newcommand{\cat}{\, \| \,}
\def\str(#1){\langle #1 \rangle}
\def\substr(#1,#2,#3){#1[#2\mbox{\,:\,}#3]}
\def\toint{\procfont{int}}
\def\tostr{\procfont{str}}
\def\byte(#1){[#1]}

% - Boolean operators
\newcommand*\AND{\wedge}
\newcommand*\OR{\vee}
\newcommand*\NOT{\neg}
\newcommand*\IMPLIES{\implies}
\newcommand*\XOR{\mathbin{\oplus}}
\newcommand*\xor{\XOR}

% - Asymptotics
\newcommand{\negl}{\procfont{negl}}
\newcommand{\poly}{\procfont{poly}}

% - Probablity
\newcommand{\E}{\mathrm{E}}
\newcommand{\Prob}[1]{\Pr\hspace{-1pt}\left[\,#1\,\right]}
\newcommand{\given}{\mid}

% Games
\newcommand{\halt}{\bot}
\newcommand{\game}{\cryptofont{G}}
\newcommand{\G}{\game}
\newcommand{\foreach}[3]{$\text{for }#1 \gets #2\text{ to }#3\text{ do}$}
\newcommand{\tab}{\hspace*{10pt}}
\newcommand{\outputs}{=}
\newcommand{\outs}{\outputs}
\newcommand{\sets}{\,\mathrm{sets}\,}
\newcommand{\bad}{\varfont{bad}}
\newcommand{\true}{1}
\newcommand{\false}{0}
\newcommand{\invalid}{\bot}
\newcommand{\exception}{\invalid}
\newcommand{\experimentv}[1]{\underline{#1}}
\newcommand{\oraclev}[1]{\underline{{oracle} #1}:}
\newcommand{\adversaryv}[1]{\underline{{adv.} #1}:}
\newcommand{\algorithmv}[1]{\underline{{alg.} #1}:}

% - Inline comment
\definecolor{CommentColor}{RGB}{125,175,230}
\newcommand{\comment}[1]{\textcolor{CommentColor}{\,\textbf{\#}\,#1}}

\newcommand{\gamesfontsize}{\footnotesize}
\newcommand{\gamespadleft}{\hskip 1pt}
\newcommand{\gamespad}{\hskip 4pt}

% - One game
\newcommand{\oneCol}[2]{
\begin{center}
        \framebox{
        \begin{tabular}{c@{\hspace*{.4em}}}
        \begin{minipage}[t]{#1\textwidth}\gamesfontsize #2 \end{minipage}
        \end{tabular}
        }
\end{center}
}

\newcommand{\twoCols}[3]{
  \makebox[\textwidth][c]{
    \begin{tabular}{|@{\gamespadleft}l@{\gamespad}|@{}@{\gamespad}l@{\gamespad}|}
    \hline
    \rule{0pt}{1\normalbaselineskip}
    \begin{minipage}[t]{#1\textwidth}\gamesfontsize
      #2 \vspace{6pt}
    \end{minipage} &
    \begin{minipage}[t]{#1\textwidth}\gamesfontsize
      #3 \vspace{6pt}
    \end{minipage} \\
    \hline
  \end{tabular}
  }
}

\newcommand{\twoColsUnbalanced}[4]{
  \makebox[\textwidth][c]{
    \begin{tabular}{|@{\gamespadleft}l@{\gamespad}|@{}@{\gamespad}l@{\gamespad}|}
    \hline
    \rule{0pt}{1\normalbaselineskip}
    \begin{minipage}[t]{#1\textwidth}\gamesfontsize
      #3 \vspace{6pt}
    \end{minipage} &
    \begin{minipage}[t]{#2\textwidth}\gamesfontsize
      #4 \vspace{6pt}
    \end{minipage} \\
    \hline
  \end{tabular}
  }
}

\newcommand{\twoColsNoDivide}[3]{
  \makebox[\textwidth][c]{
    \begin{tabular}{|@{\gamespadleft}l@{\gamespad}@{}@{\gamespad}l@{\gamespad}|}
    \hline
    \rule{0pt}{1\normalbaselineskip}
    \begin{minipage}[t]{#1\textwidth}\gamesfontsize
      #2 \vspace{6pt}
    \end{minipage} &
    \begin{minipage}[t]{#1\textwidth}\gamesfontsize
      #3 \vspace{6pt}
    \end{minipage} \\
    \hline
  \end{tabular}
  }
}

\newcommand{\twoColsNoDivideUnbalanced}[4]{
  \makebox[\textwidth][c]{
    \begin{tabular}{|@{\gamespadleft}l@{\gamespad}@{}@{\gamespad}l@{\gamespad}|}
    \hline
    \rule{0pt}{1\normalbaselineskip}
    \begin{minipage}[t]{#1\textwidth}\gamesfontsize
      #3 \vspace{6pt}
    \end{minipage} &
    \begin{minipage}[t]{#2\textwidth}\gamesfontsize
      #4 \vspace{6pt}
    \end{minipage} \\
    \hline
  \end{tabular}
  }
}

\newcommand{\twoColsTwoRows}[5]{
  \makebox[\textwidth][c]{
  \begin{tabular}{|@{\gamespadleft}l@{\gamespad}|@{}@{\gamespad}l@{\gamespad}|}
    \hline
    \rule{0pt}{1\normalbaselineskip}
    \begin{minipage}[t]{#1\textwidth}\gamesfontsize
      #2 \vspace{6pt}
    \end{minipage} &
    \begin{minipage}[t]{#1\textwidth}\gamesfontsize
      #3 \vspace{6pt}
    \end{minipage} \\
    \hline
    \rule{0pt}{1\normalbaselineskip}
    \begin{minipage}[t]{#1\textwidth}\gamesfontsize
      #4 \vspace{6pt}
    \end{minipage} &
    \begin{minipage}[t]{#1\textwidth}\gamesfontsize
      #5 \vspace{6pt}
    \end{minipage} \\
    \hline
  \end{tabular}
  }
}

\newcommand{\threeCols}[4]{
  \makebox[\textwidth][c]{
    \begin{tabular}{|@{\gamespadleft}l@{\gamespad}|@{}@{\gamespad}l@{\gamespad}|@{}@{\gamespad}l@{\gamespad}|}
    \hline
    \rule{0pt}{1\normalbaselineskip}
    \begin{minipage}[t]{#1\textwidth}\gamesfontsize
      #2 \vspace{6pt}
    \end{minipage} &
    \begin{minipage}[t]{#1\textwidth}\gamesfontsize
      #3 \vspace{6pt}
    \end{minipage} &
    \begin{minipage}[t]{#1\textwidth}\gamesfontsize
      #4 \vspace{6pt}
    \end{minipage} \\
    \hline
  \end{tabular}
  }
}

\newcommand{\twoColsThreeRows}[7]{
  \makebox[\textwidth][c]{
  \begin{tabular}{|@{\gamespadleft}l@{\gamespad}|@{}@{\gamespad}l@{\gamespad}|}
    \hline
    \rule{0pt}{1\normalbaselineskip}
    \begin{minipage}[t]{#1\textwidth}\gamesfontsize
      #2 \vspace{6pt}
    \end{minipage} &
    \begin{minipage}[t]{#1\textwidth}\gamesfontsize
      #3 \vspace{6pt}
    \end{minipage} \\
    \hline
    \rule{0pt}{1\normalbaselineskip}
    \begin{minipage}[t]{#1\textwidth}\gamesfontsize
      #4 \vspace{6pt}
    \end{minipage} &
    \begin{minipage}[t]{#1\textwidth}\gamesfontsize
      #5 \vspace{6pt}
    \end{minipage} \\
    \hline
    \rule{0pt}{1\normalbaselineskip}
    \begin{minipage}[t]{#1\textwidth}\gamesfontsize
      #6 \vspace{6pt}
    \end{minipage} &
    \begin{minipage}[t]{#1\textwidth}\gamesfontsize
       #7 \vspace{6pt}
    \end{minipage} \\
    \hline
  \end{tabular}
  }
}

% Notes
\newcounter{notectr}[section]
\newcommand{\getnotectr}{\stepcounter{notectr}\thesection.\thenotectr}
\newcommand{\basenote}[4]{{
  \textrm{\textcolor{#1}{(\getnotectr: #2 #3: #4)}}
}}

% Uncomment to mute notes.
%\renewcommand{\basenote}[4]{\ignorespaces}

\newcommand{\note}[3]{\basenote{#1}{#2}{says}{#3}}
\newcommand{\todo}[3]{\basenote{#1}{#2}{to-do}{#3}}

\newcommand{\ignore}[1]{\if{0}#1\fi}

% macros.tex
%
% Macros for this paper. (Include build/headers.tex, then this.)

\newcommand{\client}{\mathcal{C}}
\newcommand{\server}{\mathcal{S}}
\newcommand{\auth}{\mathcal{A}}
\newcommand{\key}{\schemefont{key}}
\newcommand{\hsm}{\schemefont{hsm}}
\newcommand{\mac}{\schemefont{mac}}
\newcommand{\sctr}{\varfont{session\_ctr}}
\newcommand{\tlow}{\varfont{time\_low}}
\newcommand{\thigh}{\varfont{time\_high}}
\newcommand{\tctr}{\varfont{token\_ctr}}
\newcommand{\checksum}{\schemefont{sum}}
\newcommand{\crc}{\varfont{crc}}
\newcommand{\otp}{\varfont{otp}}

\newcommand{\stringer}[1]{{\footnotesize\color{gray}\textrm{``#1''}}}
\def\str(#1){\tostr(#1)}
\def\byte(#1){[#1]}

% Variables
\newcommand{\ct}{\varfont{ct}}

% Authors' comments.
\definecolor{darkgreen}{RGB}{50,127,0}
\newcommand{\cpnote}[1]{\note{darkgreen}{Chris}{#1}}
\newcommand{\cptodo}[1]{\todo{red}{Chris}{#1}}
\newcommand{\tsnote}[1]{\note{blue}{Tom}{#1}}
\newcommand{\review}[2]{\note{red}{Reviewer #1}{#2}}
\newcommand{\anytodo}[1]{\todo{red}{\ignorespaces}{#1}}


\ifnum\camready=0
  \usepackage[letterpaper,margin=0.75in]{geometry}
\fi

\ifnum\camready=0
  \usepackage{hyperref}
  \hypersetup{
    colorlinks=true,
    allcolors={black},
    citecolor={darkgreen},
    urlcolor={darkgreen},
    pdfborder={0 0 0},
  }
\fi


\date{\today}
\title{Building a better YubiKey}
\author{Christopher Patton}
\institute{University of Florida}

\setcounter{tocdepth}{2}

\pagestyle{plain}

\begin{document}

\maketitle

\begin{abstract}
  The aim of \emph{two-factor authentication} is to tie a user's identity to
  something they \emph{know} (their password) and something they \emph{have} so
  that the attacker requires both in order to impersonate them.
  %
  The latter is often realized using a ``hardware token,'' a physical device
  that interacts with the server to authenticate the user.
  %
  \emph{YubiKey}, the flagship product of Yubico, is a low-cost, USB-enabled
  hardware token capable of supporting a variety of industry standard protocols.
  %
  Each token also supports \emph{Yubico OTP} (``one-time password''), a simple
  protocol based on symmetric-key cryptography that is designed to work
  out-of-the-box without the need to install code on the client.
  %
  But in terms of usability, this ease-of-deployment comes at a significant
  cost: we find that a YubiKey can only be used to authenticate to one service
  securely.
  %
  This work considers means by which a YubiKey may be used securely with many
  services, while keeping deployment complexity at a minimum.
  %
  Going further, we propose a redesign of the token that would
  make the hardware platform useful for a wide variety of applications beyond
  OTP-based authentication.
  \begin{keywords}
    two-factor authentication, hardware token, YubiKey
  \end{keywords}
\end{abstract}

\section{Introduction}
A crucial point of failure in virtually all web services is the user's
password. Passwords that are easy to remember are often easy to guess, a reality
which universities, companies, and governments alike must cope with.
%
Today, token-based, two-factor authentication is, arguably, the most viable way to improve
security of users' credentials.
%
A number of companies manufacture \emph{hardware tokens} for this purpose. One
of the leaders in this industry is Yubico, whose flagship product, the
\emph{YubiKey}, supports a variety of industry standard protocols, including
OATH HOTP~\cite{rfc4226} and U2F~\cite{u2f}.

YubiKeys also provide authentication out-of-the-box via the \emph{Yubico OTP}
(``one-time-password'') protocol. It involves the \emph{YubiCloud}, a web
service provided by Yubico. When the user plugs her token into a computer's USB
port and presses a button, the token types out a string just as a keybaord
would. The 44-character string, called a \emph{one-time password} (OTP), looks
something like this:
\[
  \texttt{\textcolor{gray}{ccccccflivli}fcdgrtgkhcjdfjcbljlhvehufurhtjlg}
\]
This string is typically entered into a form on a web page; when received, the web
server sends the OTP to YubiCloud for verification.
%
The first 12 characters denote the tokens's \emph{public serial identifier}. The
remaining 32 characters encode a 128-bit string, which is the AES encryption
(under a key~$K$ stored on the token) of, among other things, the token's
\emph{private serial identifier} and a \emph{counter}, which gets incremented
each time the button is pushed.
%
YubiCloud verifies the authenticity of the OTP by decrypting it and checking (1)
that the private serial identifier matches its records and (2) that the OTP is
fresh, meaning the counter is larger than the last authentic OTP it received.
YubiCloud looks up the decryption key~$K$ using the public serial identifier
that was transmitted in the clear.

Yubico OTP has a few key advantages over competing protocols.
%
First, it works without installing any software on the client's system. The
token presents itself as a USB keyboard and hence can be used with every major
operating system without additional configuration.
%
At the same time, it provides much better security than competing symmetric-key
authentication protocols, such as the popular OATH TOTP protocol (e.g. Google
Authenticator), which involves much shorter OTPs.
%
Protocols based on public-key cryptography, such as U2F, inherently provide
better security than their symmetric-key counterparts because they do not
involve shared secrets.
%
Nevertheless, there are a number of reasons to favor symmetric-key techniques,
and Yubico OTP in particular. First of all, U2F requires client-side support, and as
of this writing, only Opera, Firefox, and Chrome support its. Another reason to
favor symmetric key solutions is that the hardware requirements are much
smaller.
%
\if{0}{
  \cpnote{Another advantage of symmetric key is that the hardware tokens can be a
  lot simpler. We don't need ANY randomness for our protocols. Does U2F need
  randomness?}
  \cpnote{Could we make the argument that symmetric key is orders of magnitude
  faster, therefore better than public key for this setting? But YubiKeys already
  do a variety of public key operations, so I'm not convinced.}
}\fi

The problem of having a shared secret may be partially overcome by using
a \emph{hardware security module} (HSM) to manage the state associated with each
token. The state is encrypted using a key that (ideally) never leaves the HSM
boundary. To validate an OTP, it and the encrypted state are given to the HSM,
which decrypts the state, validates the OTP, then updates the state.
%
Yubico manufactures a low-cost, market-grade HSM, called \emph{YubiHSM} for this
purpose. It can be used to provide an authentication service similar to
YubiCloud.

\heading{Security of Yubico OTP.}
This work aims to clarify the security that Yubico OTP provides and doesn't
provide. Towards a formal treatment, we consider its security in the adversarial
model of Bellare-Rogaway for entity authentication~\cite{bellare1993entity}.
%
In this model, the adversary is assumed to utterly control the network and so is
responsible for delivering messages between players in the protocol.
%
It may inject, delay, reorder, replay, and drop messages at will.
%
The players are the set of \emph{clients}, the set of \emph{servers} requesting
authentication of the clients, and the \emph{authority} that verifies OTPs
(e.g., YubiCloud).
%
Perhaps unintuitively, we also model the set of \emph{tokens} (i.e. the YubiKeys)
managed by the authority as players in the protocol. (Hence, the adversary is
also responsible for ``delivering'' the OTP to the client.) This allows us to
capture the idea that the protocol is ``initiated'' by the client physically
interacting with the token.

We find no inherent flaws in the protocol when each YubiKey is used to
authenticate to \emph{at most one server}. In fact, we prove under standard
assumptions about AES that the probability that an adversary is able to forge an OTP to
the authority is roughly birthday bounded.
%
However, if a YubiKey is used with more than one server, than Yubico OTP
\emph{does not suffice for security in the Bellare-Rogaway model}.
%
In particular, an OTP intended for authenticating the client to one server can
be intercepted by the adversary and rerouted to another service that accepts the
same token as a credential.

\heading{OTP1.}
On the positive side, we suggest a simple modification to the protocol that
offers a defense against rerouting attacks. In our protocol, the private
identity of the token is replaced with the identity of the service requesting
authentication, allowing us to cryptographically bind the service to the OTP.
%
The protocol requires no modification to the token itself, and adds no
additional overhead (computational, communication, or bandwidth). However, it
does require a small amount of client-side code. This is unavoidable, since in order to
bind the OTP to the service, it is necessary to send the token a message. First
the client's system must be capable of talking to smart cards. (We find that
both Ubuntu 16.04 and MacOS X Sierra do so without any additional
configuration; we didn't test on Windows.)
%
\if{0}{
  \cpnote{I'm not sure about Windows.}
}\fi
%
Second, it requires the installation of a small amount of code.

\heading{OTP2.}
We also take the liberty of rethinking the design of the YubiKey altogether.
%
Ours results so far point to a fundamental axiom for token-based authentication:
either the OTP or the token itself must be bound to the service requesting
authentication. The latter is preferable, since it allows us to use a single
token for many services.
%
In practice, this means that the token will require some client-side support.
While we're at it, why restrict ourselves to OTPs?
%
We propose a modified token that supports a number \emph{modes of operation},
including simple OTP-based authentication, request-bounded OTP-based
authentication (the implicit goal of OTP1), message authentication, and
encryption.
%
We also consider a few concrete applications. For example, we show a way to provide
two-factor authentication secure against dictionary attacks on the client's
password with the same round complexity as the simple OTP protocol and with very
little computational overhead. We also suggest an extension to TLS that provides
client authentication via hardware tokens.
%
Our proposal will require modifying the YubiKey firmware at a minimum. We
suspect, however, that it can be implemented without modifying the hardware.

\heading{Organization of this report.}
Section 2 presents related work.
%
Section 3 describes the tokens (YubiKey and YubiHSM) and their functionalities.
%
Section 4 describes the OTP protocol and its security. Section 5 presents OTP1,
and section 6 presents OTP2 and applications.

\section{Background and prior work on YubiKeys}
% related.tex
%
%
\label{sec:related}

Despite their many pitfalls, passwords have long been understood as the most
practical trade-off between security and usability for authenticating users of
computer systems.
%
Early adopters of password-based authentication recognized the need to store
passwords securely~\cite{morris1979password}, yet no manner of storage is
adequate if the password can easily be guessed. This has lead system
admnistrators to enforce policies for selecting passwords (for example, adding
special characters, or requiring a minimum length) and periodically changing
them.  In their landmark user study, Adams and Sasse~\cite{adams1999users} point
out that, paradoxically, these policies \emph{adversely} affect security,
because they drive users to choose weak passwords, or even to write them down.
In light of this finding, the usability of any technique to enhance
password-based authentication has become a first-class concern.

A number of approaches to supplement (or supplant) passwords have been proposed,
each having unique barriers to adoption. Bonneau et al.~\cite{bonneau2012quest}
point out that \emph{deployability} of the mechanism is a critical factor.
%
In recent years, two-factor authentication (2FA) has emerged as an approach with
a reasonable trade-off between security, usability, and deployability.
Typically the second factor is realized using soemthing in possession of the
user, such as a cell phone or a dedicated hardware token.
%
The simplest protocol works as follows. The server associates a unique secret to
each client. When the client requests a log in, the server uses the secret and
current time to generate a short sequence (6-8) of digits, which it sends to the
client's phone. Both this sequence and the password are needed to log in. Each
``one-time password'' (OTP) is valid only for a short time (typically 30
seconds). While this system is easy to deploy, it suffers from a major security
flaw; the OTP is transmitted via SMS, an insecure channel with a number of known
vulnerabilities~\cite{reaves2016sending}. (For example, vulnerabilities in
SS7~\cite{engel2008locating} have been exploited to intercept OTPs in 2FA
systems used for bank accounts~\cite{schwachstelle}.)

An improvement came with a pair of protocols by OATH, the Initiative for Open
Authentication. The first, TOTP~\cite{rfc6238}, works as above, except the
client's phone (or token) shares a key with the server. This change affectively
removes the SMS attack vector.
%
TOTP enjoys wide adoption; a prominent example is the Google Authenticator app
for Android, iOS and other platforms~\cite{googleauth}. The second,
HOTP~\cite{rfc4226}, is a variant of TOTP that uses a counter instead of a
timestamp.
%
An early hardware token employing a protocol similar to TOTP was RSA's
SecurID~\cite{securid}.
%
Yubico OTP is similar to HOTP, in that it generates one-time passwords using a
stateful counter; however, it has several advantages from a security standpoint.
(For one, Yubico OTPs are much longer, and so harder to forge.)

Each of these protocols, including Yubico OTP, has an important drawback; they
require that the authenticating party keeps a secret. Indeed, in 2011, RSA
SecureID's server was breached, leading to the exposure of a number of tokens'
secrets~\cite{kaminsky2011securid}. This lead Yubico to develop a low-cost,
hardware-security module (the YubiHSM) in order to simplify the deployment of a
secure authentication server.
%
Of courese, public-key cryptography affords the opportunity to avoid this
drawback altogether. The FIDO Allience, in collaboration with a number of
industry leaders, has developed the universal two-factor (U2F)
protocol~\cite{u2f}, which usess a private key stored on a hardware token; the
corresponding pulbic key is stored on the server.
%
YubiKeys support this more sophisticated protocol in addition to Yubico OTP. The
latter has the distinct advantage of being easier to deploy; U2F requires the
web browser to support it, since it involves a couple rounds of communications
between the token and the server. Yubico OTP requires no browser support.

Due to its simplicity, Yubico OTP remains an important industry player. As such,
it has received a respectable amount of attention in academia.
%
The first formal tratment of the full protocol was provided by K\"unnemann and
Steel~\cite{kuennemann2012yubisecure}; however, their result is limited in a
fundamental way. They work in a restricted adversarial model, called the
\emph{Dolev-Yao} model~\cite{herzog2005computational}. Roughly speaking, they
prove that no Dolev-Yao adversary seeing an unbounded number of OTPs can recover
the underlying secret. This is a weaker security property than we generally hope
for. In particular, the stronger \emph{Bellare-Rogaway}
model~\cite{bellare1993entity} directly captures an adversary's ability to forge
the users's credentials.  From this perspective, the exact security of Yubico
OTP remains open.

K\"unneman and Steel also pointed out attacks on the YubiHSM that, with a
particular configuration, allows an attacker on the authority's system to
decrypt sate associated with YubiKeys.
%
YubiKey has also been shown to be susceptible to side-channel attacks. Oswald et
al.~\cite{oswald2013side-channel} use minimally-invasive power analysis to
recover the token's secret key for generating OTPs.

\ignore{
\begin{cool}
\cpnote{Maybe cite
\url{http://ieeexplore.ieee.org/document/4625610/?arnumber=4625610}}
\end{cool}
}



\section{YubiKey and YubiHSM}
\label{sec:tokens}
This section provides an overview of the functionalities exposed by the hardware
tokens, namely the YubiKey (denoted \key) and the YubiHSM (denoted \hsm).
%
Before we begin, we note that the protocol involves three principals: the
\emph{client} being authenticated, the \emph{server} requesting
authentication of the client, and the \emph{authority} that authenticates the
client to the server. The client possesses~\key and the authority possesses~\hsm.

\heading{YubiKey 4.}
%
There are four varieties of tokens: two are have a large profile designed to be
carried by the client, and two have a small profile designed to remain in the
client's machine. Yubico manufactures a USB-A and USB-C variant for each profile
type.
%
All four varieties support the same suite of protocols.

The token has two (re)programmable slots. A slot is configured to run one of a
variety of protocols, including Yubico OTP, OATH-HOTP, challenge-response
(either OTP- or HMAC-based), or it can be programmed to output a static
password. The protocol is activated by pressing (and holding) on the key. The
first slot is activated by pressing the button and immediately letting go; the
second slot is activated by pressing and holding for a few seconds.
%
All of the values stored on the token, including the shared secret, are
generated by the programmer and written to the device. The programmer can choose
to disable configuration; from that point on, security-critical values cannot be
overwritten.
%
\cpnote{Can a slot configured for, say, OTP-based CR, be used in a different
way, say HMAC-based CR?}

The token presents three interfaces to the operating system: the first two are
\emph{human interface devices} (HIDs) and the third is a \emph{smart card}. The
two HIDs correspond to the two slots and are used for the non-interactive
protocols, in particular Yubico OTP, OATH-HOTP, and static password. The
challenge-response protocols require the client to send a message to the device;
this is handled by the smart card interface.
%
\cptodo{Double check with Dave.}

In this work we are interested in OTP and OTP-based challenge-response.
We write execution of the OTP protocol as $(\otp, \sigma') \gets
\key(\stringer{otp}, \sigma)$, where $\otp$ denotes the OTP, $\sigma$ denotes
the key's internal state, and $\sigma'$ denotes the token's update internal
state. (We emphasize that the client gets the OTP, but not the token's state.)
%
Execution of OTP challenge-response is written $(Y, \sigma') \gets
\key(\stringer{chal}, m, \sigma)$, where $m$ is a 6-byte string called the
\emph{challenge}, and $Y$ is a 16-byte string called the \emph{response}.

\heading{YubiHSM 1.6.}
%
The HSM is designed to provide secure management of shared secrets in protocols
involving YubiKeys, but it also offers a few other cryptographic functionalities
that make it useful for other applications. A properly configured YubiHSM can be
used to improve security. The client needn't trust the authority to keep a
secret; she need only trust that the HSM is properly configured.
%
\cpnote{The configuration of the YubiHSM constitutes an intricate attack
surface, one that warrants further study.}
%
All operations are based on
symmetric cryptography. (However, the newer YubiHSM 2, described below, also
supports a few public-key operations.)
%
The token provides the operating system with two serial interfaces: one for
configuration, and the other for executing protocols.
%
\cptodo{Double check this with Dave.}

The token is configured to operate in one of two modes: \emph{HSM} and
\emph{WSAPI}. The latter facilitates easy deployment of Yubico OTP for a limited
number of keys, while the latter offers more features. Changing the mode of
operation requires resetting the token, and resetting requires physical access.
(It also requires physical access to configure the device.)

\textit{HSM.} This mode offers a number of features, including OTP validation,
management of shared secrets, and a handful of generic operations.
%
Each operation is associated with a \emph{key handle} for a key generated by and
stored within the token. (The HSM holds up to 40 keys.) Also associated to each
key is a set of \emph{permitted operations} using that key. All keys 128-bit
strings generated using the HSM's random number generator.
%
\cpnote{The key store itself is ``encrypted using AES-256'' in non-volatile
memory. Where is does the key come from for encrypting the key store?''}
%
Each key can be configured to perform the following operations:

\begin{itemize}
  \item \textit{Key management.} The HSM can be used to wrap keys for external
    storage. Keys can then be unwrapped within the HSM to be used as a
    \emph{temporary key} (accessed with special handle). AES-CCM is used for key
    (un)wrapping.
    %
    \cpnote{It would be interesting to see how Tom's crypto API
    paper~\cite{SSW16} applies to YubiHSM key wrapping.}

  \item \textit{External Yubico OTP validation.} The HSM stores the key, public and private
    identifiers, and counter state associated with each YubiKey it manages. The
    state is stored externally, using AES-CCM is used for confidentiality and
    integrity.

  \item \textit{Internal Yubico OTP validation.} The HSM can store the state of
    up 1024 YubiKeys in its internal database.

  \item \textit{Message authentication.} A key handle can be used to authenciate
    a message using HMAC-SHA1.

  \item \textit{``Plain'' encryption/decryption.} A key may be used in AES-ECB mode.
    %
    \cpnote{This amounts to a blockcipher oracle. If a key that is used for
    AES-CCM is also used for AES-ECB mode, then ciphertext can easily be
    decrypted using an attack by~\cite{kuennemann2012yubisecure}. The HSM is
    still vulnerable to this attack, since a key may be configured to work in
    both modes. I wonder what other ``bad'' configurations there are.}
\end{itemize}
Another feature of the HSM is \textit{pseudorandom number generation}. Entropy
is gathered from timing difference between input/output events. The PRNG can
also be reseeded using entropy provided by the user. The PRNG is also used to
generate keys stored on the HSM.
%
\cpnote{Dave is interested in chocking the entropy of the PRNG. I'm wondering
what algorithm is used.}

\textit{WSAPI.}
%
The operations above can be used to implement the Yubico OTP protocol as
described in Section~\ref{sec:otp} without exposing any sensitive values to the
authority.
%
The WSAPI mode\footnote{See
\url{https://developers.yubico.com/yubikey-val/Validation_Protocol_V2.0.html}.}
allows the protocol to be implemented conveniently for a limited number of
YubiKeys.
%
This is the ``mainstream'' mode of operation for the HSM, as it supports the
Yubico OTP protocol with a minimal attack surface. A major limitation, however,
is that this mode stores all secrets internally; thus, it is only useful for a
limited deployment.

\heading{YubiHSM 2.}
%
Recently, Yubico released its new version of the YubiHSM, which offers a number
of new features. It supports the common crypto API standards Microsoft CNG and
PKCS\#11, in addition to the native API above.
%
\cpnote{We should check that the API is actually the same.}
%
It also supports new crypto operations, such SHA2, RSA- and EC-based
signing/encryption, and attestation of the HSM's state and configuration.


\section{The Yubico OTP protocol}
\label{sec:otp}
%
This section specifies the Yubico OTP protocol in the manner in which it is
generally deployed.\footnote{See
\url{https://developers.yubico.com/OTP/Specifications}.}
%
In order to simplify the exposition, we will not assume the use of a YubiHSM;
note, however, that the HSM could be used to keep all sensitive values hidden from
the authority, effectively reducing trust in the authority to trusting that the
HSM has been properly configured.
%
We begin with a bit of notation.

\heading{Notation.}
If~$n$ is a positive integer, let $[1..n]$ denote the set of integers from~$1$ to~$n$.
%
Variables are strings unless noted otherwise.
%
Strings are finite sequences of bytes, i.e., elements of $(\bits^8)^*$, unless
noted otherwise. Let~$\emptystr$ denote the empty string.
%
Let~$X$ and~$Y$ be strings and let $X \cat Y$ denote their concatenation.
%
String indexing has the same semantics as the Python programming language: let
$0 \leq i \leq j \leq |X|-1$. Then $X[i]$ denotes $(i+1)$-th byte of~$X$ and
$\substr(X,i,j)$ denotes the sub string $X[i] \cat \cdots \cat X[j-1]$ of $X$.
Let $\substr(X,i,) = \substr(X,i,|X|-1)$ and $\substr(X,,j) = \substr(X,0,j)$.
%
If $i \geq j$, then $\substr(X,i,j) = \emptystr$ by convention.
%
Let $\str(X)$ denote the modhex encoding of~$X$.\footnote{OTPs
use the alphabet \texttt{cbdefghijklnrtuv} instead of the usual
\texttt{0123456789abcdef} for encoding nibbles.}
%
Let $\byte(i) \in \bits^8$ denote the byte encoding of integer $i\in[0..255]$.

\heading{The data frame.}
The \key stores a 16-byte key~$K$. When activated, it emits a 44-byte string,
called a \emph{one-time password}.
%
An OTP is a modhex string $\str(I \cat Y)$, where $|I|=6$ and $|Y|=16$.
%
String $I$ is the \emph{public identifier} of \key and $Y=E_K(X)$, where~$E$ denotes the
AES-128 block cipher and $X$ is a 16-byte string called the \emph{data frame}.
The data frame encodes the following quantities:
\begin{itemize}
  \item string $m$ ($\substr(X,,6)$) --- the payload.
  \item integer $\sctr$ ($\substr(X,6,8)$) --- the session counter.
    Changes each time $\tctr$ rolls over.
  \item integer $\tlow$ ($\substr(X,8,10)$) --- the low-order time stamp
    bits. Changes quickly.
  \item integer $\thigh$ ($X[10]$) --- the high-order time stamp
    bits. Changes slowly.
  \item integer $\tctr$ ($X[11]$) --- the token counter. Changes each
    time \key is invoked.
  \item string $r$ ($\substr(X,12,14)$) --- a pseudorandom value generated by
    \key. \cpnote{How is this seeded? Can this be reseeded?}
  \item integer $\crc$ ($\substr(X,14,)$) --- a checksum of
    $\substr(X,,14)$, denoted $\checksum(\substr(X,,14))$.\footnote{Computed as \texttt{0xffff} minus the output of
    \url{https://github.com/Yubico/yubico-c/blob/master/ykcrc.c}.}
\end{itemize}
The payload is the \emph{private identifier} of \key in the Yubico OTP protocol.
%
An OTP is deemed \emph{authentic} if the payload of the decrypted frame matches
the private ID the authority has on record \emph{and} the checksum is correct.
An OTP is \emph{fresh} if the counter is larger than the previous frame. To
enable this functionality, it suffices to have the authority store the frame of
the last authentic and fresh OTP it received. The \emph{initial frame} is
defined by setting the payload to the private ID, setting the counters, time
stamp, and pseudorandom bytes all to zero, and computing the checksum of the
frame.
%
Define $X.\ct$ as $X.\tctr + (X.\sctr \ll 8)$, where the $\ll$ dentos the bit
shift operator.

\heading{The protocol.}
Let $\client$ denote the client in possession of \key, $\server$ denote the
server, and~$\auth$ denote the authority.
%
The server and authority share a key~$K'$ for HMAC-SHA1, denoted $\mac$. The
protocol works as follows:
%
\begin{enumerate}
  \item $\client$ executes $(\otp, \sigma') \gets \key(\stringer{otp}, \sigma)$
    and transmits $\otp$ to $\server$.

  \item $\server$ computes $H = \mac_{K'}(J \cat N \cat \otp)$ and sends $(H, J,
    N, \otp)$ to $\auth$, where~$N$ is a nonce and~$J$ is $\server$'s identity.

  \item $\auth$ looks up the key $K'$ associated with~$J$ and checks that $H =
    \mac_{K'}(J \cat N \cat \otp)$. If so, it proceeds to step~4; otherwise it
    halts.

  \item $\auth$ decodes $\str(I\cat Y) \gets \otp$ and looks up the key~$K$ and
    data frame~$T$ associated with~$I$.
    %
    It computes $X \gets E_K^{-1}(Y)$, checks that $\otp$ is authentic (that $X.m =
    T.m$ and $\checksum(\substr(X,,14))=X.\crc$) and fresh (that $X.\ct > T.\ct$).
    %
    If both conditions hold, it lets $T \gets X$.  It lets~$R$ denote the result
    of check. Finally, it computes $H' = \mac_{K'}(N \cat R \cat \otp)$, and sends $(H',
    N, R, \otp)$ to $\server$.

  \item $\server$ checks that $H' = \mac_{K'}(N \cat R \cat \otp)$ and that the
    response~$R$ says that $\otp$ is authentic and fresh. If so, it accepts;
    otherwise it rejects.
\end{enumerate}

\textit{Initialization.}
The key~$K$ and initial frame~$T$ are generated and used to compute the $\key$'s
initial state~$\sigma$, and~$K$ and~$T$ are given to~$\auth$.
%
Next, key~$K'$ is generated and given to $\auth$ and $\server$.
%
The public identity~$I$ of $\key$ is given to~$\server$, and the identity~$J$ of
$\server$ is given to~$\auth$.

\textit{Verification of~$H$ is optional.}
%
Computing the MAC of the server request in step 2 and verification of MAC in
step 3 is an optional feature of the protocol. This step is not essential for
authenticating the client, but may be useful for other purposes.

\textit{Interpreting the result $R$.}
%
The contents of~$R$ are specified by the protocol.\footnote{See
\url{https://developers.yubico.com/OTP/Specifications/OTP_decryption_protocol.html}.}
If the $\otp$ is valid, then it includes the counter and the time stamp.
Otherwise it indicates what went wrong; for example, the $\otp$ was mal-formed,
inauthentic, or not fresh.

\subsection{Security}
We describe, informally, the goal of the Yubico OTP protocol.
%
Fix $c,s,t\in\N$.  The players are the set of clients $\{\client_i\}_{i\in[c]}$,
the set of servers $\{\server_j\}_{j\in[s]}$, the set of
tokens~$\{\key_k\}_{k\in[t]}$, and the authority~$\auth$.
%
A client may be in possession of any number of tokens, but no token is possessed by
more than one client.
%
Clients ``register'' at most one token with each server, meaning the server is
given the public identity of the token.
%
A client may register any token with any server, and a single token may be used for
any number of servers.

\heading{Threat model.}
We adopt the threat model of~\cite{bellare1993entity}. The adversary is
active and controls all flows between players. We assume that the players are
initialized (as described above) before the attack begins.
%
Instances of the protocol, called \emph{sessions}, may be carried out
simultaneously.
%
Each session is initialized by the adversary and is defined by a triple
of positive integers $(i, j, k)$.
%
A \emph{valid session} is one in which the Yubico OTP protocol involving
$\client_i$, $\server_j$, $\key_k$, and $\auth$ is carried out faithfully, and
where $\client_i$  is ``in possession of'' $\key_k$.
%
A session is not valid unless it completes.
%
A session is called \emph{accepting} if $\server_j$ accepts in that session.
%
The goal of the adversary is to get \emph{any} server to accept in an
\emph{invalid} session. We say the protocol is secure if the probability that
any ``reasonable'' adversary can do so is negligible.

\textit{Physical assumptions.}
In our adversarial model, we assume that the tokens expose the interface
described in Section~\ref{sec:tokens}, and otherwise offer no attack surface.
%
Of course, this is a strong assumption that does not stand up to scrutiny. For
example, prior work has shown YubiKeys to be susceptible to side-channel
attacks~\cite{oswald2013side-channel}. Thus, an adversary in control of a
client's system, or in physical proximity of the token, may have access to
additional information not captured in the model.
%
\cpnote{This might make an interesting attack surface, but I'll ignore it for
the time being.}

In the following, we consider the various avenues of attack available to our
adversary.

\heading{Forging the authority response.}
The use of HMAC-SHA1 and the nonce~$N$ in the server response ensure that the
response is authentic and cannot be replayed. (Each server must take
care to make sure that it never repeats a nonce.) The MAC also binds the response
to the OTP, ensuring that if the OTP is valid, then the response is valid. In
the remainder, we focus on the validity of the OTP.

\newcommand{\forge}{\notionfont{FORGE}}
\newcommand{\sprp}{\notionfont{SPRP}}
\newcommand{\advF}{\algorithmfont{F}}
\newcommand{\advD}{\algorithmfont{D}}
\newcommand{\calK}{\mathcal{K}}
\newcommand{\calE}{\mathcal{E}}
\newcommand{\win}{\varfont{win}}
\newcommand{\KEYO}{\oraclefont{Key}}
\newcommand{\AUTHO}{\oraclefont{Auth}}
\newcommand{\Perm}{\mathrm{Perm}}
\newcommand{\Dom}{\mathrm{Dom}\,}
\newcommand{\Rng}{\mathrm{Rng}\,}
\begin{figure}[t]
  \newcommand{\isvalid}{\schemefont{isvalid}}
  \twoCols{0.48}
  {
    \underline{$\Exp{\forge}_{E,S}(\advF)$}\\[2pt]
      $\calE \gets \emptyset$;
      $K \getsr \calK$;
      $T \gets S$;
      $\win \gets \false$;
      $\advF^{\,\KEYO,\AUTHO}(S)$\\
      return $\win$
    \\[6pt]
    \underline{$\KEYO(\,)$}\\[2pt]
      $S.\ct \gets S.\ct + 1$;
      $Y \gets E_K(S)$;
      $\calE \gets \calE \union \{Y\}$;
      return $Y$
    \\[6pt]
    \underline{$\AUTHO(Y)$}\\[2pt]
      $X \gets E_K^{-1}(Y)$\\
      if $\isvalid(X,T)$ then\\
      \tab if $Y \not\in \calE$ then $\win \gets \true$\\
      \tab $T \gets X$\\
      return $X$\\[2pt]
    \hrule
    \vspace{5pt}
    \underline{$\Exp{\sprp}_{E,b}(\advD)$}\\[2pt]
      $K \getsr \calK$; $\pi \getsr \Perm(n)$\\
      if $b=1$ then $b' \getsr \advD^{E_K(\cdot),E_K^{-1}(\cdot)}$\\
      else $b' \getsr \advD^{\pi(\cdot),\pi^{-1}(\cdot)}$\\
      return $b'$
  }
  {
    \underline{$\G^c_S(\advF)$}\\[2pt]
      $\calE \gets \emptyset$;
      $T \gets S$;
      $\win \gets \false$;
      $\advF^{\,\KEYO,\AUTHO}(S)$\\
      return $\win$
    \\[6pt]
    \underline{$\KEYO(\,)$}\\[2pt]
      $S.\ct \gets S.\ct + 1$\\
      if $S \not\in \Dom \pi$ then $\pi(S) \getsr \bits^n \setminus \Rng \pi$\\
      $\calE \gets \calE \union \{\pi(S)\}$;
      return $\pi(S)$
    \\[6pt]
    \underline{$\AUTHO(Y)$}\\[2pt]
      if $Y \in \calE$ then $X \gets \pi^{-1}(Y)$\\
      \tab if $\isvalid(X,T)$ then $T \gets X$\\
      \tab return $X$\\
      $X \getsr \bits^n$\\
      if $X \in \Dom \pi$ then\\
      \tab $\bad \gets \true$; if $c=1$ then $X \getsr \bits^n \setminus \Dom \pi$\\
      $\pi^{-1}(Y) \gets X$\\
      if $\isvalid(X,T)$ then\\
      \tab $\win \gets \true$; $T \gets X$\\
      return $X$
  }
  \caption{Let $E:\calK\cross\bits^n\to\bits^n$ be a blockcipher, $\calK$ be a
    finite set, and $n\in\N$. Let $\Perm(n)$ denote the set of all permutations
    over $\bits^n$. Let $\isvalid(X,T)$ denote the predicate $(X.m = T.m) \AND (X.\crc =
    \checksum(\substr(X,,14))) \AND (X.\ct > T.\ct)$.
  \textbf{Top-left:} the \forge game for~$E$;
  \textbf{Bottom-left:} the \sprp game for~$E$; and
  \textbf{Right:} a game used to fix Lemma 1.}
  \label{fig1}
  \vspace{6pt}
  \hrule
\end{figure}
\heading{Forging an OTP.}
A successful forgery is an attack whereby the adversary transmits an OTP to a
server that the authority deems authentic and fresh, but was not output by the
token.
%
An OTP is merely the AES encryption of some ``stuff'' using a key shared between
each $\key_k$ and $\auth$, and so it is natural to reduce an adversary's ability
to forge to the security of AES.
%
Typically we would assume that AES is a good \emph{pseudorandom permutation},
meaning that no ``reasonable'' adversary can distinguish the output of the
blockcipher from a true random permutation, even if the adversary chooses the
inputs.
%
However, this does not suffice for our setting, since the forgery adversary has,
in effect, an oracle for the permutation \emph{and} its inverse. A more
appropriate assumption is that AES a is a \emph{strong pseudorandom
permutation}.

Let $n\in\N$, $\calK$ be a finite set, and let $E:\calK\cross\bits^n\to\bits^n$
be a blockcipher. We associated to $E$, an adversary~$\advD$, and bit~$b$ a
game, \sprp, defined in Figure~\ref{fig1}.
%
If $b=1$, then a key~$K$ is chosen uniformly from the set~$\calK$ and~$\advD$ is
given access to oracles~$E_K(\cdot)$ and $E_K^{-1}(\cdot)$.
%
Otherwise, a permutation $\pi$ is chosen uniformly from the set of all
permutations over~$\bits^n$ and the adversary is given oracles for~$\pi$ and its
inverse.
%
Eventually it outputs a bit~$b'$, which is returned by the game. We define the
advantage of~$\advD$ in attacking~$E$ as
\[
  \Adv{\sprp}_E(\advD) = \left| \Prob{\Exp{\sprp}_{E,1}(\advD)\outputs\true} -
                                \Prob{\Exp{\sprp}_{E,0}(\advD)\outputs\true}
                              \right|\,.
\]
Informally, we call~$E$ a strong pseudorandom permutation if
$\Adv{\sprp}_E(\advD)$ is ``small'' for all ``reasonable''~$\advD$.

We note that there is a significant gap between PRP and \sprp security. On the
theoretical side, there are constructions of blockciphers that are provably
secure as PRPs, but are \emph{not} SPRPs, e.g. the Feistel constructions
of~\cite{luby1988how}.
%
On the other hand, the \sprp security of AES has undergone significant
cryptanalysis, e.g. using the boomerang attack of
Wanger~\cite{wagner1999boomerang}. It is generally believed (and often assumed)
that AES-128 is a secure \sprp.

\textit{One client, one token, and one server.}
Under the \sprp assumption, we can prove a useful lemma that helps us quantify the
adversary's ability to forge an OTP. Suppose for the moment that there is just
one client with a single token registered with one server.
%
We may formalize the forgery attack as the game \forge defined in
Figure~\ref{fig1}. The game is parameterized by an \emph{initial data
frame}~$S$, which specifies the shared state between the token and authority
before the attack begins.
The adversary~$\advF$ is given an oracle~$\KEYO$ for the
token and an oracle~$\AUTHO$ for authenticating OTPs.
%
The game sets a flag~$\win$ if the
adversary makes an $\AUTHO$-query that is fresh and authentic, but was never
output by~$\KEYO$. The outcome of the game is the value of~$\win$ when~$\advF$
halts. We define the advantage of~$\advF$ in forging against~$E$ (beginning at
state~$S$) as
\[
  \Adv{\forge}_{E,S}(\advF) = \Prob{\Exp{\forge}_{E,S}(\advF)\outputs\true} \,.
\]

\heading{Lemma 1. }\emph{Let $n=128$, $\calK$ be a finite set, and
  $E:\calK\cross\bits^n\to\bits^n$ be a blockcipher.
  Let $S \in \bits^n$ be a data frame and let $\tau = S.\ct < 2^{24}$.
  Let $\advF$ be a \forge adversary making~$\kappa<2^{24}$ queries to~$\KEYO$
  and~$\alpha$
  queries to~$\AUTHO$.
  There exists an \sprp adversary~$\advD$ such that
  \[
    \Adv{\forge}_{E,S}(\advF) \leq
      \Adv{\sprp}_E(\advD)
      + \frac{\alpha^2+2\kappa\alpha}{2^{129}}
      + \frac{\alpha(2^{24}-\tau)}{2^{88}}
  \]
  where~$\advD$ has the same runtime as~$\advF$.
}\\[2pt]
\textit{Proof.}
%
Adversary~$\advD$ is constructed by simulating~$\advF$ in its game in the
natural way. In particular, calls to the blockcipher (i.e. the call to~$E$
in~$\KEYO$) are forwarded to~$\advD$'s first oracle, and calls to the inverse of
the blockcipher (i.e. the call to~$E^{-1}$ in~$\AUTHO$) are forwarded
to~$\advD$'s second oracle. Once~$\advF$ halts, adversary~$\advD$ halts and
outputs~$\true$ if $(\win=\true)$ and~$\false$ otherwise.

Let~$b$ denote the challenge bit in~$\advD$'s game. If $b=1$, then the
simulation is perfect; otherwise ~$\advD$'s output is identically distributed
to the output of $\G^1_S(\advF)$, where game~$\G^1$ is as defined in
Figure~\ref{fig1}. (This game is the same as the \forge game instantiated with a random
permutation rather than a blockcipher. The permutation is simulated via lazy
evaluation.) It follows that
\begin{eqnarray*}
  \Adv{\sprp}_E(\advD) &=& \Prob{\Exp{\sprp}_{E,1}(\advD)\outputs\true} -
                           \Prob{\Exp{\sprp}_{E,0}(\advD)\outputs\true}\\
                       &=& \Prob{\Exp{\forge}_{E,S}(\advF)\outputs\true} -
                           \Prob{\G^1_S(\advF)\outputs\true}\,.
\end{eqnarray*}

We next note the random variables $\G^1_S(\advF)$ and $\G^0_S(\advF)$ are
identically distributed until either game sets the flag~$\bad\gets\true$.
This occurs if~$\advF$ asks $\AUTHO(Y)$, where~$Y$ was never output by~$\KEYO$,
and~$X$ chosen uniformly from~$\bits^n$ is in the domain of~$\pi$. On the $i$-th
query to $\AUTHO$ This occurs with probability at most $(\kappa + i - 1)/2^n$.
Summing over all~$\alpha$ queries yields
\[
  \left| \Prob{\G^1_S(\advF)\outputs\true} - \Prob{\G^0_S(\advF)\outputs\true}
  \right| \leq \frac{2\kappa\alpha + \alpha^2}{2^{n+1}}\,.
\]

We need only bound $\Prob{\G^0_S(\advF)\outputs\true}$. In this game, each query
to~$\AUTHO(Y)$ such that $Y$ was never output by~$\KEYO$ samples an~$X$ from
$\bits^n$ independently of all previous queries. Hence, the probability that
that $\win$ gets set, i.e. $X.m = T.m$, $X.\crc=\checksum(\substr(X,,14))$, and
$X.\ct > T.\ct$, is independent for each query.
%
The payload ($X.m$) is the first 48 bits (6 bytes), the checksum $(X.\crc)$ is
the last 16 bits (2 bytes), and the counter $(X.\ct)$ is comprised of 24  bits
(3 bytes). Since these values are non-overlapping, there are at most
$2^{n-88}(2^{24}-\tau)$ values for~$X$ that would set~$\win$. (The precise
number depends on the number of OTPs ``delivered'' by the adversary, i.e. the
number of queries $\AUTHO(Y)$ such that~$Y$ was output by $\KEYO$.) Since~$X$ is
a uniform random string, it follows
that
\[
  \Prob{\G^0_S(\advF)\outputs\true} \leq \frac{\alpha(2^{24}-\tau)}{2^{88}}\,.
\]
This yields the final bound.
%
\qed

\textit{Interpreting the bound.}
%
The Yubico OTP protocol achieves roughly birthday-bound (i.e., 64-bit, half
the block length) security in the \forge game, meaning the bound becomes vacuous
after the adversary makes about $\alpha=2^{64}$ forgery attempts. (Or somewhat
less, depending on the value of ~$\kappa$, which is the number of OTPs the
adversary gets to see.)
%
The last term in the bound gets smaller as the initial counter value
increases, but is at most~$\alpha/2^{64}$ (i.e., when $\tau=0$).
%
Of course, this bound should not be interpreted as the concrete security for the
overall protocol; it only captures one attack vector. However, we can conclude
that the concrete security of the overall protocol is no better than this.

\heading{Rerouting an OTP.}
Lemma 1 captures an adversary's ability to forge an OTP when there is just one
client, one token, and one server. More generally, suppose there are any number
of clients, tokens, and servers, but each token is used to authenticate the
client to at most one server. It's not difficult to see that security in this
more general setting reduces to \forge security. (Perhaps with a small cost in
concrete security.)
%
However, if we allow a client to register a single token with more than
one server, then there is a trivial attack whereby the adversary simply
\emph{reroutes} an OTP meant for one server to another.
%
Suppose that $\client_i$ has registered $\key_k$ with servers $\server_j$ and
$\server_{j^*}$.
%
The adversary initiates a session $(i,j,k)$ and runs it until $\key_k$ outputs
$\otp$ to $\client_i$.
%
It then initiates a new session $(i,j^*,k)$, but instead of faithfully running
the protocol, it sends $\otp$ to $\server_{j^*}$, then runs the protocol from that
point forward. In the end, $\server_{j^*}$ will accept, although the session is
invalid.

\heading{Conclusion.}
%
The Yubico OTP protocol appears to provide strong (at least birthday-bound)
security as long as a client never uses the same token for more than one
service, since an adversary that can intercept an OTP can easily use it to
authenticate to another service.
%
A na\"ive solution would be to require the client and server to use TLS so that
the OTP is not sent in the clear. But unless the client and server are mutually
authenticated during the handshake, the channel is vulnerable to a
man-in-the-middle attack. Since the OTP is itself a means of authentication,
this is not a viable solution.

\textit{The private identifier is useless.}
The \forge game is parameterized by the initial data frame, which models the
shared state between the token and authority in the protocol.  The data frame is
given to the adversary in the game, Hence, the private identifier being secret
is not essential for security. It suffices to have a public identifier and a
secret key associated with it. As Lemma~1 indicates, the payload can be
anything, for example, the all zero string. Indeed, many of the components of
the data frame can be removed without compromising security: in particular, the
checksum, the pseudorandom value, and time stamp. We only need the payload (some
publicly-known value) and the counter.


\section{OTP1}
\label{sec:otp1}
In this section, we consider means by which a YubiKey can be used securely for
multiple services. The goal will be to bind the OTP to the service requesting
authentication.
%
We first recall that each YubiKey has two slots, which both can be configured to
run Yubico OTP with different services. However, we will aim for a protocol that
can be used with any number of services.
%
Unfortunately, any such solution will involve \emph{sending a message} to the
token, which means we must forfeit a major advantage of Yubico OTP: the protocol
requires minimal client-side support, since the client's system must only be
capbble of recognizing a USB keybaord. Our solution will require the client to
install a small amount of code,\footnote{See
\url{https://github.com/Yubico/python-yubico}.} and their system must be capable
of talking to smart cards.\footnote{Both Ubuntu 16.04 LTS and Mac OS 10.12.6
(``Sierra'') are able to talk to the YubiKey without issue.}

\heading{OTP challenge-response.}
In Section~\ref{sec:tokens} we briefly described the \emph{challenge-response}
mode of the YubiKey. In more detail, the server sends to the client a short
(6-byte), random challenge. The client forwards this to their YubiKey, which
transforms it into a data frame with the challenge as its payload. It outputs
the encrypted data frame, which is forwarded to the server, then to the
authority. The authority decrypts and checks that it is fresh and that the
checksum bits match; if so, it sends the payload to the server. Finally, the
server checks that the payload matches the challenge.

\newcommand{\hash}{\schemefont{hash}}
The challenge-response protocol above is sufficient to mitigate rerouting
attacks as long as no two servers choose the same challenge. (Assuming these are
chosen randomly, a collision is unlikely.) However, it adds an extra 0.5 round
trip and requires the server to maintain a bit of extra state.
%
We suggest the following alternative.

\heading{The revised protocol.} The client \emph{hashes the identity of
the service}, and uses the first 6 bytes of the hash as the challenge. Let
$\hash$ be a cryptographic hash function whose output is at least~6 bytes in
length. The
new protocol is as follows:
\begin{enumerate}
  \item Given $J$, the identity of~$\server$, the client $\client$ computes $m
    \gets \substr({\hash(J)},6,)$ and executes $(Y, \sigma') \gets
    \key(\stringer{chal}, m, \sigma)$. It then transmits $\otp=\str(I\cat Y)$ to
    $\server$.
    %
    \cpnote{Not sure how to obtain~$I$ from the YubiKey.}

  \item $\server$ computes $H = \mac_{K'}(J \cat N \cat \otp)$ and sends $(H, J,
    N, \otp)$ to $\auth$, where~$N$ is a nonce and~$J$ is $\server$'s identity.

  \item $\auth$ looks up the key $K'$ associated with~$J$ and checks that $H =
    \mac_{K'}(J \cat N \cat \otp)$. If so, it proceeds to step~4; otherwise it
    halts.

  \item $\auth$ decodes $\str(I\cat Y) \gets \otp$ and looks up the key~$K$ and
    data frame~$T$ associated with~$I$.
    %
    It computes $X \gets E_K^{-1}(Y)$, checks that $\otp$ is authentic (that $X.m =
    \substr({\hash(J)},6,))$ and $\checksum(\substr(X,,14))=X.\crc$) and fresh (that $X.\ct > T.\ct$).
    %
    If both conditions hold, it lets $T \gets X$.  It lets~$R$ denote the result
    of check. Finally, it computes $H' = \mac_{K'}(N \cat R \cat \otp)$, and sends $(H',
    N, R, \otp)$ to $\server$.

  \item $\server$ checks that $H' = \mac_{K'}(N \cat R \cat \otp)$ and that the
    response~$R$ says that $\otp$ is authentic and fresh. If so, it accepts;
    otherwise it rejects.
\end{enumerate}
Initialization is the same as in the standard Yubico OTP protocol
(Section~\ref{sec:otp}), except that~$\client$ is also given the identity~$J$
of~$\server$.

\subsection{Security}
%
The changes to the protocol are minimal, and so we expect it to have the same
security properties as Yubico OTP.
%
In addition, it offers a defense against OTP rerouting. Suppose that $J$ and
$J^*$ are the identity of two servers and $\substr({\hash(J)},6,) \ne
\substr({\hash(J^*)},6,)$. Then an OTP generated in a session for $(i, j, k)$
will not be deemed authentic in a session for $(i, j^*, k)$.

This begs the question: how likely is it that the payloads for two server
identities collide? We \emph{cannot} bound this probability using the collision
resistance of $\hash$, since we need to truncate its output quite a bit in order
to fit it in the frame. (For example, SHA1 outputs 20 bytes and SHA256 outputs
32, whereas the payload is just 6 bytes.)
%
In the random-oracle model, we can argue that the probability is at most
$s^2/2^{48}$, where~$s$ is the number of services. But even this bound is
significantly weaker than what we would hope for. In particular, it is much
weaker than the bound in Lemma~1.

Still, this change to the protocol is a viable option for allowing deployed
YubiKeys to be used with multiple servers securely. But what if we're willing to
modify the tokens? In the next section, we propose a way to significantly
improve security. We suspect our proposal will require changing the firmware,
but not the YubiKey hardware.

\textit{No YubiHSM support for this protocol.}
Note that the YubiHSM 1.6 does not support OTP challenge-response.  Recall that
in the Yubico OTP protocol, the expectation is that the payload remain private.
But in the challenge-response protocol, it is necessary that the payload be sent
to the server. In our protocol, the authority can check that the payload matches
what is expected, but this does not appear to be supported.
%
\cpnote{What about YubiHSM 2?}


\section{OTP2}
\newcommand{\tsprp}{\widetilde{\notionfont{SPRP}}}
\newcommand{\calT}{\mathcal{T}}
\newcommand{\ENCO}{\oraclefont{E}}
\newcommand{\DECO}{\oraclefont{E}^{-1}}
Co-opting the challenge-response mode as described above is a viable way to
use YubiKeys with multiple servers, but there is certainly room for improvement.
%
One way to improve security is to modify the frame data structure so that the
payload is longer. The only essential component is the 3-byte counter, leaving
13 bytes for the payload.
%
However, as long as we're modifying the YubiKey, we can do much better. In the
following, we present a revised OTP protocol, called \emph{OTP2}, that uses a
\emph{tweakable blockcipher} as a building block. Our design may be used in
multiple \emph{modes of operation}, supporting a wide variety of applications
beyond OTP-based authentication.

\newcommand{\otpsec}{\notionfont{IND\$\mbox{-}OTP}}
\newcommand{\OTPO}{\oraclefont{OTP}}
\newcommand{\mode}{\varfont{mode}}
\newcommand{\OTP}{\schemefont{otp}}
\newcommand{\INIT}{\schemefont{init}}
\newcommand{\calM}{\setfont{M}}
\begin{figure}
  \newcommand{\fmtframe}{\schemefont{frame}}
  \twoColsUnbalanced{0.38}{0.58}
  {
    \underline{$\Exp{\tsprp}_{E,b}(\advD)$}\\[2pt]
      $K \getsr \calK$;
      $b' \getsr \advD^{\,\ENCO,\DECO}$\\
      return $b'$
    \\[6pt]
    \underline{$\ENCO(T, X)$}\\[2pt]
      if $b=1$ then return $E_K(T, X)$\\
      if $\pi_T = \bot$ then $\pi_T \getsr \Perm(n)$\\
      return $\pi_T(X)$
    \\[14pt]
    \underline{$\DECO(T, Y)$}\\[2pt]
      if $b=1$ then return $E^{-1}_K(T, Y)$\\
      if $\pi_T = \bot$ then $\pi_T \getsr \Perm(n)$\\
      return $\pi^{-1}_T(Y)$
  }
  {
    \underline{$\Exp{\otpsec}_{E,b}(\advD)$}\\[2pt]
       $K \getsr \calK$; $\ct \gets 0$; $\calE \gets \emptyset$;
       $b' \getsr \advD^{\,\OTPO,\OTPO^{-1}}$\\
       return $b'$
    \\[6pt]
    \underline{$\OTPO(T, \mode, m)$}\\[2pt]
      $\ct \gets \ct + 1$;
      $X \gets \fmtframe(\mode, \ct, m)$
        \comment{Outputs an OTP2 frame.}\\
      if $b=1$ then $Y \gets E_K(T, X)$\\
      else $Y \getsr \bits^n$\\
      $\calE \gets \calE \union \{(T, X, Y)\}$; return $Y$
    \\[6pt]
    \underline{$\OTPO^{-1}(T,Y)$}\\[2pt]
      if $(\exists X)\,(T, X, Y) \in \calE$ then return $X$\\
      if $b=1$ then $X \gets E^{-1}_K(T, Y)$\\
      else $X \getsr \bits^n$\\
      $\calE \gets \calE \union \{(T, X, Y)\}$; return $X$
  }
  \caption{\textbf{Left:} $\tsprp$ security and \textbf{Right:} \otpsec security of
  tweakable blockciphers.}
  \label{fig2}
  \vspace{6pt}
  \hrule
\end{figure}
A \emph{tweakable blockcipher}~\cite{liskov2011tweakable} is a deterministic algorithm
$E:\calK\cross\calT\cross\bits^n\to\bits^n$, where $n$ is a positive integer, $\calT$ is a set,
and $\calK$ is a finite set such that for each $K\in\calK$ and $T\in\calT$,
function $E_K(T,\cdot)$ is a bijection, with $E_K^{-1}(T,\cdot)$ denoting its
inverse.
%
The goal is that each tweak induces a permutation that looks random and
independent from the permutations induced by other tweaks. Security may be
formalized by the game defined in the left-hand panel of Figure~\ref{fig2}.

A number of efficient, secure constructions of tweakable blockciphers are known.
The simplest construction, due to Liskov, Rivest, and
Wagner~\cite{liskov2011tweakable} combines a hash function and a blockcipher.
Let $h : \calT \to \bits^n$ be a function sampled from a family of
$\epsilon$-almost-xor-universal hash functions. This means that for every
$T,T'\in\calT$, where $T\ne T'$, and $\Delta\in\bits^n$, the probability that $h(T)\xor h(T') =
\Delta$ is at most $\epsilon$.
%
(There are several well-known families of hash functions having this property.)
Let $E:\calK\cross\bits^n\to\bits^n$ be a standard blockcipher.  The LRW
tweakable blockcipher is defined by $\widetilde{E}_K(T,X) = E_K(X \xor \Delta)
\xor \Delta$, where $\Delta = h(T)$.
%
XORing $h(T)$ with the input ensures that, with high probability, the inputs of
the blockcipher are different for different tweaks. The result is that
blockciphers with different tweaks look independent to a computationally-bounded
adversary.

\heading{The data frame.}
The protocol will use a tweakable blockcipher with arbitrary-length tweaks.
(Many such constructions are known, cf.~\cite{liskov2011tweakable,landescher2012tweak}.)
%
Namely, let $\calK$ be a finite set, $\calT=\bits^*$, $n\in\N$ be a multiple
of~$8$, $N=n/8$, and $E:\calK\cross\calT\cross\bits^n\to\bits^n$ be a tweakable
blockcipher. The OTP2 frame is much simpler than the OTP frame, having only
three components. It is designed to support multiple ``modes of operation'' used
for protocols beyond one-time-password-based authentication. (We will
discuss these shortly.)
\begin{itemize}
  \item integer $X.\mode$ ($X[0]$) --- Specifies the mode of operation for the
    frame. This is used to signal to the authority which protocol is being
    executed.
  \item integer $X.\ct$ ($\substr(X,1,4)$) --- The 3-byte (24-bit) counter. This
    has the same semantics as $(\tctr,\sctr)$ in the OTP protocol; it is
    incremented each time the token is engaged.
  \item string $X.m$ ($\substr(X,4,)$) --- The payload, comprised of $N-4$
    bytes.
\end{itemize}
The semantics of the payload and tweak are determined by the mode of operation; accordingly,
whether the frame is deemed authentic depends on the mode. However, its freshness is
determined by the $X.\ct$ as before, and the authority must deem a non-fresh
frame to be invalid.
%
The format of the enciphered OTP2 is much the same as before, except that it
depends on a tweak. Namely, it is computed as $\otp = \str(I \cat E_K(T,X))$,
where~$T$ is the tweak and~$I$ is the identity of the token.

\newcommand{\fancykey}{\schemefont{key2}}
\heading{The token.}
%
The OTP2 hardware token will only do one thing: encipher data frames.  The token
takes as input the tweak~$T$, an integer $\mode$, and the payload~$m$.  When
activated, it first checks that $|m| = N-4$ and $\mode$ is a valid mode. It
then increments the internal counter, computes the data frame~$X$, computes $Y
\gets E_K(T,X)$, and outputs $Y$.
%
The token may be used in two ways. The first provides the no-client-side-code
functionality of YubiKeys. Though it is inherently insecure when used for
multiple services, we feel it should be supported. The second mode supports the
full suite of OTP2-based protocols. The token has the following interfaces:
\begin{itemize}
  \item $(\otp, \sigma') \gets \fancykey(\stringer{otp0}, \sigma)$. Outputs a
    fully-formed OTP2 in $\mode=0$.
  \item $(Y, \sigma') \gets \fancykey(\stringer{otp}, T, \mode, m, \sigma)$.
    Outputs an enciphered OTP2 data frame.
  \item $I \gets \fancykey(\stringer{id}, \sigma)$. Outputs the public identity
    of the token.
\end{itemize}
%
Each of these ``calls'' use the same state, so that $(\otp, \sigma') \gets
\fancykey(\stringer{otp}, \sigma)$ is equivalent to running
\[
  (Y, \sigma) \gets \fancykey(\stringer{otp}, \emptystr, 0, \byte(0)^{N-4},  \sigma);
  I \gets \fancykey(\stringer{id}, \sigma)
\]
then computing $\otp \gets \str(I \cat Y)$. (See the OTP0 mode below.)

\textit{Physical considerations.}
An interface is \emph{activated} either by the arrival of some inputs or by the
client physically interacting with it.
%
The \stringer{otp} interface is activated as follows. When a payload is waiting,
the token starts blinking. Once the user \emph{engages} the token by pressing
the button, the payload is processed and an enciphered frame is output. Not every
mode of operation requires engagement.
%
The \stringer{otp0} interface is activated when there is no payload waiting and
when the client engages the token.
%
The \stringer{id} may be activated without engagement.
%
\cpnote{Should the token somehow enable ``cancelation?''}

\subsection{Modes of operation}
We describe four modes of operation for the token.
%
\cpnote{This is the most interesting part from a cryptographic perspective. All
of this looks ``OK'' to me, but each definitely needs a rigorous analysis.}

\heading{OTP0 (\textnormal{$\mode=0$}).}
%
The simple, one-time-password mode. The payload consists of an all-zero byte
string $\byte(0)^{N-4}$ and the tweak is $\emptystr$. Upon receipt, the
authority deciphers the frame, checks that the payload is equal to
$\byte(0)^{N-4}$, and checks that the counter is fresh. This is the
``no-client-side-code'' mode that can be invoked via the token's \stringer{otp0}
interface. \emph{It must be used with only one service}, and must be engaged by
the client.

\heading{OTP (\textnormal{$\mode=1$}).}
%
The \emph{request-bounded, one-time-password} mode. Provides authentication of the
client for a particular request. The request might be to login to a service, or
for authorization to access a resource. The resource is encoded by the tweak.
The payload is specified by the protocol, but it is a static value known to the
client and authority. Upon receipt, the authority decrypts and checks that the
payload matches the string it expects.
%
Requires engagement.

\heading{Integrity (\textnormal{$\mode=2$}).} The \emph{integrity} mode. This
can be used to authenticate (i.e. MAC) a message. The payload is
$\byte(0)^{N-4}$ and the tweak is the message.
%
Requires engagement.

The next two modes are used to implement \emph{authenticated encryption with
associated data}. Such a scheme provides privacy and authenticity of a
message~$M$ and authenticity for associated data~$A$. Normally the syntax
requires an explicit nonce, but our scheme will not require a nonce and take
advantage of the token's stateful counter. It resembles the OCB mode of
operation, but has some significant differences.
%
The string~$M\cat\byte(1)\cat\byte(0)^p$, where $p$ is the smallest, positive
integer such that $|M| + 1 + p \equiv 0 \pmod{N-4}$, is divided into
($N-4$)-byte blocks $(M_1, M_2, \ldots, M_\ell)$.  These are XORed together to
get the checksum~$R$.
%
Each block is processed in order in $\mode=3$ (defined below) to get $(C_1, C_2,
\ldots, C_\ell)$. These blocks are XORed together to get~$S$. Then
$R \xor \substr(S,,N-4)$ is processed in $\mode=4$ to get~$T$.  Finally, the
ciphertext $C_1 \cat C_2 \cat \cdots C_\ell \cat T$ is transmitted to the
authority.
%
The authority decrypts each ciphertext block as it arrives, checking that the
counter value is larger for each block. (The counters need not be contiguous.)
The last block is used to determine if the message is authentic. The authority
must output the message only if it is authentic. The adversary deciphers the
block (using~$A$ as the tweak), and checks that that the payload matches the
checksum of the ciphertext and message blocks.
%
\cpnote{Worth comparing this to https://eprint.iacr.org/2013/835.pdf.}

\heading{Transport (\textnormal{$\mode=3$}).}
%
Processes a message block in transport mode. The payload is the message
block~$M_i$ and the tweak is~$\emptystr$. This mode \emph{does not} require
engagement.

\heading{Transport Finish (\textnormal{$\mode=4$}).}
%
Process the checksum of the message in transport mode. The payload is the
truncated checksum $\substr({(R\xor S)},4,)$ and the tweak is the associated
data~$A$. Requires engagement.
%
\if{0}{
  \cpnote{This is the major departure from OCB. In that mode, $A$ is processed
  in PMAC-fashion and added to the tag, and the tweak is derived from the nonce
  and a counter (starting at 0). We don't need to tweak in this way because we
  stuff a counter into the BC call.}
}\fi

\subsection{Security}
The OTP2 is a building block for four modes of operation: simple authentication,
request-bounded authentication, integrity, and transport. Each mode has a
different security goal, and the hope is that the token is a powerful enough
tool to achieve them.
%
Of course, we need a proof of security for each mode. To that end, we have
formulated a notion of security that models the token's operation on the client's
side and the processing of the output on the authority's side.
%
The \otpsec game (right-hand side of Figure~\ref{fig2}) is associated to a
tweakable blockcipher, a bit~$b$, and an adversary~$\advD$. The adversary is
asked to distinguish the output of the token from a random string given
given access to an oracle for the token's \stringer{otp} interface.  On input of
a tweak, mode, and payload, if $b=1$, then the oracle increments a counter,
constructs and enciphers a frame using the provided tweak, and returns the
output. If $b=0$, it outputs a randomly chosen string. It is also given an
oracle for deciphering frames. If $b=1$, then it outputs the deciphered frame;
otherwise it returns a random string.

Intuitively, security in the \otpsec sense implies that the OTP output of the
token may be treated as a uniform random string. On the other hand, if the
authority is asked to decipher an OTP not output by the token --- or perhaps the
OTP is valid, but the query involves the wrong tweak --- then the deciphered
frame will look random. We conjecture that these properties suffice to prove
security of each of the modes of operation.
%
\cpnote{This might be harder to prove than I think. AFAIK, OTP2 constitutes a
\textbf{new security model}. The OTP2 is being used for different purposes
simultaneously. I haven't seen any other crypto primitive used quite as flexibly
as this.}
%
However, we first need to prove the following:

\heading{Conjecture 1.} \emph{The $\tsprp$ security of tweakable
blockcipher~$E$ implies the \otpsec security of~$E$.}

\noindent
We suspect the proof follows from a similar argument used in Lemma~1.
If so, we should get roughly birthday-bound security.


\subsection{Applications}
\label{sec:apps}
We present a few interesting applications in order to motivate the deployment of
OTP2.

\newcommand{\id}{\varfont{id}}
\newcommand{\pw}{\varfont{pw}}
\heading{Two-factor authentication secure against dictionary attacks.}
%
\emph{OTP mode} can be used to avoid sending a password hash in the clear via the
following protocol. Let $\hash$ be a cryptographic hash function with output
length of $N-4$ bytes. The client~$\client$ has a token~$\fancykey$ managed by
authority~$\auth$ and a username-password pair $(\id, \pw)$. The
server~$\server$ knows the identity~$I$ of~$\fancykey$ and $H = \hash(\id \cat
\pw)$.
%
\begin{enumerate}
  \item $\client$ choses a random, $(N-4)$-byte string~$R$.
    %
    It then computes $(Y, \sigma) \gets \fancykey(\stringer{otp}, 1, R
    \xor H, J, \sigma)$, where~$J$ is the identity of the service~$\server$.
    It then computes $I \gets \fancykey(\stringer{id}, \sigma)$, $\otp \gets
    \str(I \cat Y)$, and sends $(R, \otp, \id)$ to~$\server$.

  \item $\server$ looks up the hash~$H$ associated with~$\id$, then computes $V
    = H\xor R$ and sends $(J, V, \otp)$ to~$\auth$.

  \item $\auth$ decodes $\str(I \cat Y) \gets \otp$, looks up the key~$K$ and
    counter~$\ct$ associated with~$I$, then computes $X \gets E^{-1}_K(J, Y)$.
    %
    It checks if~$X.\ct > \ct$ and $X.m = V$: if so, it lets $\ct \gets X.\ct$
    and sends $(J, \otp, \stringer{ok})$ to~$\server$; otherwise it sends $(J,
    \otp, \stringer{invalid})$ to~$\server$.

  \item If~$\auth$'s response says \stringer{ok}, then accept; otherwise reject.
\end{enumerate}
Each message sent between $\server$ and $\auth$ is encrypted and authenticated
using their shared key. (For example, they may use TLS.) This differs from
the OTP (Section~\ref{sec:otp}) and OTP1 (Sectoin~\ref{sec:otp1}) protocols,
which only require these messages to be authenticated.

This protocol has the same round complexity as OTP and OTP1 and incurs only a
bit of extra overhead.
%
A key advantage of this protocol is that~$H$ is never sent in the clear, which
prevents dictionary attacks on the password by~$\auth$.  Dictionary attacks by a
network adversary are prevented by using a secure channel (i.e. authenticated
encryption) between~$\server$ and~$\auth$.

\heading{Fine-grained resource authorization.}
%
Typically, web services have a \emph{course-grained} model of authorization,
meaning once you've provided credentials and logged in, you needn't provide your
credentials again until you've logged out.
%
There are notable exceptions, however. For example, GitHub requires you to
provide your credentials before deleting a repository. To take another example,
Amazon sometimes requires you to re-enter the last four digits of your credit
card number in order to make a purchase. These exceptions motivate a need for
\emph{finer-grained} authorization for access to extra sensitive resources or
to take certain actions.
%
This is a perfect application for \emph{OTP mode}.\footnote{Imagine it: you're
about to delete a GitHub repo, and the website prompts you to stick in your
token. Then a message pops up: ``Are you sure you want to delete this?'' Click
``yes'': ``Engage your token to confirm.''} The OTP1 protocol might be used for
the same purpose. However, the tiny, 6-byte payload used their means collisions
are quite likely.

\heading{Client authentication extension to TLS.}
%
In a typical TLS handshake, the server is authenticated to the client, but not
the other way around. This leaves open the possibility of a man-in-the-middle.
This looks like a job for \emph{integrity mode!}
%
Suppose the client and server are using ephemeral Diffie-Hellman. The
ClientHello message contains the client's key share $g^a$. The client computes
$(T, \sigma) \gets \fancykey(\stringer{otp}, 2, g^a, \sigma)$, then $I \gets
\fancykey(\stringer{id}, \sigma)$, and sends~$\str(I \cat T)$ along with the
ClientHello.
%
The server forwards $g^a$ and $\str(I \cat T)$ to the authority, who computes $X \gets
E^{-1}_K(g^a, T)$ and checks that $X$ is fresh and $X.m = \byte(0)^{N-4}$. If
so, it updates its state and tells the server to accept.

\heading{Key rotation.}
%
The \emph{transport mode} provides a secure channel between the client and the
authority. One way this can be used is to securely update the token's
configuration. Currently, the only way to update the shared secret is to
generate a fresh secret and physically copy it to the client and authority's
systems. The transport mode could be used to \emph{wrap} a key generated on the
client's system, then transmit the wrapped key to the authority. The same
technique can be used to update the counter, the public identity, and any other
state that needs to be shared between the token and authority.

\subsection{Concrete instantiations of the tweakable blockcipher}

Our requirements for the underlying primitive are an infinite tweak space (i.e.,
$\calT=\bits^*$) and a reasonable block size (e.g., 16 bytes). There are a
number of constructions that would suit our needs, but we need to carefully
consider the constraints of the token's hardware platform. YubiKeys already
support AES-128 and HMAC-SHA1. It should be easy to show, under appropriate
assumptions, that
\[
  \widetilde{E}_{K\cat K'}(T, X) = E_K(X \xor \Delta) \xor \Delta,
  \text{where}\, \Delta = \substr({H_{K'}(T)},,16)
\]
is $\tsprp$ secure, where~$E$ denotes AES-128 and~$H$ denotes HMAC-SHA1. The
assumptions are that~$E$ is an \sprp and $H$ is a PRF. This is essentially the
LRW construction; the $\epsilon$-almost-xor-universality is provided by the PRF
security of~$H$.

Since we're using algorithms already supported by the token, we suspect that
OTP2 can be deployed without modifying the YubiKey hardware; only changes to the
firmware should be required. One potential issue is that the secret key is now
twice as long, since $|K \cat K'| = 32$. (HMAC-SHA1 calls for a
16-byte key.)
%
\cpnote{It depends on how the key is stored, I suppose. If it's impossible store
larger keys, then perhaps we can design a new TBC that uses only a 16-byte key.}
%
It's also worth noting that HMAC-SHA1 should not be considered a secure
PRF, since SHA1 is known to be broken. (We have found a collision!) Therefore,
it may be worth replacing SHA1 with SHA256.

\textit{Precomputing~$\Delta$.}
%
The transport and OTP0 modes use $\emptystr$ as the tweak. The value of
$\Delta_\emptystr = \substr({H_{K'}(\emptystr)},,16)$ might be precomputed and
stored on the token, thus avoiding an HMAC-SHA1 evaluation for these modes.


\ifnum\camready=0
 \bibliographystyle{alpha}
 \else
\bibliographystyle{build/splncs_srt}
\fi
\bibliography{yubi}

\end{document}
