A crucial point of failure in virtually all web services is the user's
password. Passwords that are easy to remember are often easy to guess, a reality
which universities, companies, and governments alike must cope with.
%
Today, token-based, two-factor authentication is, arguably, the most viable way to improve
security of users' credentials.
%
A number of companies manufacture \emph{hardware tokens} for this purpose. One
of the leaders in this industry is Yubico, whose flagship product, the
\emph{YubiKey}, supports a variety of industry standard protocols, including
OATH HOTP~\cite{rfc4226} and U2F~\cite{u2f}.

YubiKeys also provide authentication out-of-the-box via the \emph{Yubico OTP}
(``one-time-password'') protocol. It involves the \emph{YubiCloud}, a web
service provided by Yubico. When the user plugs her token into a computer's USB
port and presses a button, the token types out a string just as a keybaord
would. The 44-character string, called a \emph{one-time password} (OTP), looks
something like this:
\[
  \texttt{\textcolor{gray}{ccccccflivli}fcdgrtgkhcjdfjcbljlhvehufurhtjlg}
\]
This string is typically entered into a form on a web page; when received, the web
server sends the OTP to YubiCloud for verification.
%
The first 12 characters denote the tokens's \emph{public serial identifier}. The
remaining 32 characters encode a 128-bit string, which is the AES encryption
(under a key~$K$ stored on the token) of, among other things, the token's
\emph{private serial identifier} and a \emph{counter}, which gets incremented
each time the button is pushed.
%
YubiCloud verifies the authenticity of the OTP by decrypting it and checking (1)
that the private serial identifier matches its records and (2) that the OTP is
fresh, meaning the counter is larger than the last authentic OTP it received.
YubiCloud looks up the decryption key~$K$ using the public serial identifier
that was transmitted in the clear.

Yubico OTP has a few key advantages over competing protocols.
%
First, it works without installing any software on the client's system. The
token presents itself as a USB keyboard and hence can be used with every major
operating system without additional configuration.
%
At the same time, it provides much better security than competing symmetric-key
authentication protocols, such as the popular OATH TOTP protocol (e.g. Google
Authenticator), which involves much shorter OTPs.
%
Protocols based on public-key cryptography, such as U2F, inherently provide
better security than their symmetric-key counterparts because they do not
involve shared secrets.
%
Nevertheless, there are a number of reasons to favor symmetric-key techniques,
and Yubico OTP in particular. First of all, U2F requires client-side support, and as
of this writing, only Opera, Firefox, and Chrome support its. Another reason to
favor symmetric key solutions is that the hardware requirements are much
smaller.
%
\if{0}{
  \cpnote{Another advantage of symmetric key is that the hardware tokens can be a
  lot simpler. We don't need ANY randomness for our protocols. Does U2F need
  randomness?}
  \cpnote{Could we make the argument that symmetric key is orders of magnitude
  faster, therefore better than public key for this setting? But YubiKeys already
  do a variety of public key operations, so I'm not convinced.}
}\fi

The problem of having a shared secret may be partially overcome by using
a \emph{hardware security module} (HSM) to manage the state associated with each
token. The state is encrypted using a key that (ideally) never leaves the HSM
boundary. To validate an OTP, it and the encrypted state are given to the HSM,
which decrypts the state, validates the OTP, then updates the state.
%
Yubico manufactures a low-cost, market-grade HSM, called \emph{YubiHSM} for this
purpose. It can be used to provide an authentication service similar to
YubiCloud.

\heading{Security of Yubico OTP.}
This work aims to clarify the security that Yubico OTP provides and doesn't
provide. Towards a formal treatment, we consider its security in the adversarial
model of Bellare-Rogaway for entity authentication~\cite{bellare1993entity}.
%
In this model, the adversary is assumed to utterly control the network and so is
responsible for delivering messages between players in the protocol.
%
It may inject, delay, reorder, replay, and drop messages at will.
%
The players are the set of \emph{clients}, the set of \emph{servers} requesting
authentication of the clients, and the \emph{authority} that verifies OTPs
(e.g., YubiCloud).
%
Perhaps unintuitively, we also model the set of \emph{tokens} (i.e. the YubiKeys)
managed by the authority as players in the protocol. (Hence, the adversary is
also responsible for ``delivering'' the OTP to the client.) This allows us to
capture the idea that the protocol is ``initiated'' by the client physically
interacting with the token.

We find no inherent flaws in the protocol when each YubiKey is used to
authenticate to \emph{at most one server}. In fact, we prove under standard
assumptions about AES that the probability that an adversary is able to forge an OTP to
the authority is roughly birthday bounded.
%
However, if a YubiKey is used with more than one server, than Yubico OTP
\emph{does not suffice for security in the Bellare-Rogaway model}.
%
In particular, an OTP intended for authenticating the client to one server can
be intercepted by the adversary and rerouted to another service that accepts the
same token as a credential.

\heading{OTP1.}
On the positive side, we suggest a simple modification to the protocol that
offers a defense against rerouting attacks. In our protocol, the private
identity of the token is replaced with the identity of the service requesting
authentication, allowing us to cryptographically bind the service to the OTP.
%
The protocol requires no modification to the token itself, and adds no
additional overhead (computational, communication, or bandwidth). However, it
does require a small amount of client-side code. This is unavoidable, since in order to
bind the OTP to the service, it is necessary to send the token a message. First
the client's system must be capable of talking to smart cards. (We find that
both Ubuntu 16.04 and MacOS X Sierra do so without any additional
configuration; we didn't test on Windows.)
%
\if{0}{
  \cpnote{I'm not sure about Windows.}
}\fi
%
Second, it requires the installation of a small amount of code.

\heading{OTP2.}
We also take the liberty of rethinking the design of the YubiKey altogether.
%
Ours results so far point to a fundamental axiom for token-based authentication:
either the OTP or the token itself must be bound to the service requesting
authentication. The latter is preferable, since it allows us to use a single
token for many services.
%
In practice, this means that the token will require some client-side support.
While we're at it, why restrict ourselves to OTPs?
%
We propose a modified token that supports a number \emph{modes of operation},
including simple OTP-based authentication, request-bounded OTP-based
authentication (the implicit goal of OTP1), message authentication, and
encryption.
%
We also consider a few concrete applications. For example, we show a way to provide
two-factor authentication secure against dictionary attacks on the client's
password with the same round complexity as the simple OTP protocol and with very
little computational overhead. We also suggest an extension to TLS that provides
client authentication via hardware tokens.
%
Our proposal will require modifying the YubiKey firmware at a minimum. We
suspect, however, that it can be implemented without modifying the hardware.

\heading{Organization of this report.}
Section 2 presents related work.
%
Section 3 describes the tokens (YubiKey and YubiHSM) and their functionalities.
%
Section 4 describes the OTP protocol and its security. Section 5 presents OTP1,
and section 6 presents OTP2 and applications.

\section{Background and prior work on YubiKeys}
% related.tex
%
%
\label{sec:related}

Despite their many pitfalls, passwords have long been understood as the most
practical trade-off between security and usability for authenticating users of
computer systems.
%
Early adopters of password-based authentication recognized the need to store
passwords securely~\cite{morris1979password}, yet no manner of storage is
adequate if the password can easily be guessed. This has lead system
admnistrators to enforce policies for selecting passwords (for example, adding
special characters, or requiring a minimum length) and periodically changing
them.  In their landmark user study, Adams and Sasse~\cite{adams1999users} point
out that, paradoxically, these policies \emph{adversely} affect security,
because they drive users to choose weak passwords, or even to write them down.
In light of this finding, the usability of any technique to enhance
password-based authentication has become a first-class concern.

A number of approaches to supplement (or supplant) passwords have been proposed,
each having unique barriers to adoption. Bonneau et al.~\cite{bonneau2012quest}
point out that \emph{deployability} of the mechanism is a critical factor.
%
In recent years, two-factor authentication (2FA) has emerged as an approach with
a reasonable trade-off between security, usability, and deployability.
Typically the second factor is realized using soemthing in possession of the
user, such as a cell phone or a dedicated hardware token.
%
The simplest protocol works as follows. The server associates a unique secret to
each client. When the client requests a log in, the server uses the secret and
current time to generate a short sequence (6-8) of digits, which it sends to the
client's phone. Both this sequence and the password are needed to log in. Each
``one-time password'' (OTP) is valid only for a short time (typically 30
seconds). While this system is easy to deploy, it suffers from a major security
flaw; the OTP is transmitted via SMS, an insecure channel with a number of known
vulnerabilities~\cite{reaves2016sending}. (For example, vulnerabilities in
SS7~\cite{engel2008locating} have been exploited to intercept OTPs in 2FA
systems used for bank accounts~\cite{schwachstelle}.)

An improvement came with a pair of protocols by OATH, the Initiative for Open
Authentication. The first, TOTP~\cite{rfc6238}, works as above, except the
client's phone (or token) shares a key with the server. This change affectively
removes the SMS attack vector.
%
TOTP enjoys wide adoption; a prominent example is the Google Authenticator app
for Android, iOS and other platforms~\cite{googleauth}. The second,
HOTP~\cite{rfc4226}, is a variant of TOTP that uses a counter instead of a
timestamp.
%
An early hardware token employing a protocol similar to TOTP was RSA's
SecurID~\cite{securid}.
%
Yubico OTP is similar to HOTP, in that it generates one-time passwords using a
stateful counter; however, it has several advantages from a security standpoint.
(For one, Yubico OTPs are much longer, and so harder to forge.)

Each of these protocols, including Yubico OTP, has an important drawback; they
require that the authenticating party keeps a secret. Indeed, in 2011, RSA
SecureID's server was breached, leading to the exposure of a number of tokens'
secrets~\cite{kaminsky2011securid}. This lead Yubico to develop a low-cost,
hardware-security module (the YubiHSM) in order to simplify the deployment of a
secure authentication server.
%
Of courese, public-key cryptography affords the opportunity to avoid this
drawback altogether. The FIDO Allience, in collaboration with a number of
industry leaders, has developed the universal two-factor (U2F)
protocol~\cite{u2f}, which usess a private key stored on a hardware token; the
corresponding pulbic key is stored on the server.
%
YubiKeys support this more sophisticated protocol in addition to Yubico OTP. The
latter has the distinct advantage of being easier to deploy; U2F requires the
web browser to support it, since it involves a couple rounds of communications
between the token and the server. Yubico OTP requires no browser support.

Due to its simplicity, Yubico OTP remains an important industry player. As such,
it has received a respectable amount of attention in academia.
%
The first formal tratment of the full protocol was provided by K\"unnemann and
Steel~\cite{kuennemann2012yubisecure}; however, their result is limited in a
fundamental way. They work in a restricted adversarial model, called the
\emph{Dolev-Yao} model~\cite{herzog2005computational}. Roughly speaking, they
prove that no Dolev-Yao adversary seeing an unbounded number of OTPs can recover
the underlying secret. This is a weaker security property than we generally hope
for. In particular, the stronger \emph{Bellare-Rogaway}
model~\cite{bellare1993entity} directly captures an adversary's ability to forge
the users's credentials.  From this perspective, the exact security of Yubico
OTP remains open.

K\"unneman and Steel also pointed out attacks on the YubiHSM that, with a
particular configuration, allows an attacker on the authority's system to
decrypt sate associated with YubiKeys.
%
YubiKey has also been shown to be susceptible to side-channel attacks. Oswald et
al.~\cite{oswald2013side-channel} use minimally-invasive power analysis to
recover the token's secret key for generating OTPs.

\ignore{
\begin{cool}
\cpnote{Maybe cite
\url{http://ieeexplore.ieee.org/document/4625610/?arnumber=4625610}}
\end{cool}
}

