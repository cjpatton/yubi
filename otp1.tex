\label{sec:otp1}
In this section, we consider means by which a YubiKey can be used securely for
multiple services. The goal will be to bind the OTP to the service requesting
authentication.
%
We first recall that each YubiKey has two slots, which both can be configured to
run Yubico OTP with different services. However, we will aim for a protocol that
can be used with any number of services.
%
Unfortunately, any such solution will involve \emph{sending a message} to the
token, which means we must forfeit a major advantage of Yubico OTP: the protocol
requires minimal client-side support, since the client's system must only be
capbble of recognizing a USB keybaord. Our solution will require the client to
install a small amount of code,\footnote{See
\url{https://github.com/Yubico/python-yubico}.} and their system must be capable
of talking to smart cards.\footnote{Both Ubuntu 16.04 LTS and Mac OS 10.12.6
(``Sierra'') are able to talk to the YubiKey without issue.}

\heading{OTP challenge-response.}
In Section~\ref{sec:tokens} we briefly described the \emph{challenge-response}
mode of the YubiKey. In more detail, the server sends to the client a short
(6-byte), random challenge. The client forwards this to their YubiKey, which
transforms it into a data frame with the challenge as its payload. It outputs
the encrypted data frame, which is forwarded to the server, then to the
authority. The authority decrypts and checks that it is fresh and that the
checksum bits match; if so, it sends the payload to the server. Finally, the
server checks that the payload matches the challenge.

\newcommand{\hash}{\schemefont{hash}}
The challenge-response protocol above is sufficient to mitigate rerouting
attacks as long as no two servers choose the same challenge. (Assuming these are
chosen randomly, a collision is unlikely.) However, it adds an extra 0.5 round
trip and requires the server to maintain a bit of extra state.
%
We suggest the following alternative.

\heading{The revised protocol.} The client \emph{hashes the identity of
the service}, and uses the first 6 bytes of the hash as the challenge. Let
$\hash$ be a cryptographic hash function whose output is at least~6 bytes in
length. The
new protocol is as follows:
\begin{enumerate}
  \item Given $J$, the identity of~$\server$, the client $\client$ computes $m
    \gets \substr({\hash(J)},6,)$ and executes $(Y, \sigma') \gets
    \key(\stringer{chal}, m, \sigma)$. It then transmits $\otp=\str(I\cat Y)$ to
    $\server$.
    %
    \cpnote{Not sure how to obtain~$I$ from the YubiKey.}

  \item $\server$ computes $H = \mac_{K'}(J \cat N \cat \otp)$ and sends $(H, J,
    N, \otp)$ to $\auth$, where~$N$ is a nonce and~$J$ is $\server$'s identity.

  \item $\auth$ looks up the key $K'$ associated with~$J$ and checks that $H =
    \mac_{K'}(J \cat N \cat \otp)$. If so, it proceeds to step~4; otherwise it
    halts.

  \item $\auth$ decodes $\str(I\cat Y) \gets \otp$ and looks up the key~$K$ and
    data frame~$T$ associated with~$I$.
    %
    It computes $X \gets E_K^{-1}(Y)$, checks that $\otp$ is authentic (that $X.m =
    \substr({\hash(J)},6,))$ and $\checksum(\substr(X,,14))=X.\crc$) and fresh (that $X.\ct > T.\ct$).
    %
    If both conditions hold, it lets $T \gets X$.  It lets~$R$ denote the result
    of check. Finally, it computes $H' = \mac_{K'}(N \cat R \cat \otp)$, and sends $(H',
    N, R, \otp)$ to $\server$.

  \item $\server$ checks that $H' = \mac_{K'}(N \cat R \cat \otp)$ and that the
    response~$R$ says that $\otp$ is authentic and fresh. If so, it accepts;
    otherwise it rejects.
\end{enumerate}
Initialization is the same as in the standard Yubico OTP protocol
(Section~\ref{sec:otp}), except that~$\client$ is also given the identity~$J$
of~$\server$.

\subsection{Security}
%
The changes to the protocol are minimal, and so we expect it to have the same
security properties as Yubico OTP.
%
In addition, it offers a defense against OTP rerouting. Suppose that $J$ and
$J^*$ are the identity of two servers and $\substr({\hash(J)},6,) \ne
\substr({\hash(J^*)},6,)$. Then an OTP generated in a session for $(i, j, k)$
will not be deemed authentic in a session for $(i, j^*, k)$.

This begs the question: how likely is it that the payloads for two server
identities collide? We \emph{cannot} bound this probability using the collision
resistance of $\hash$, since we need to truncate its output quite a bit in order
to fit it in the frame. (For example, SHA1 outputs 20 bytes and SHA256 outputs
32, whereas the payload is just 6 bytes.)
%
In the random-oracle model, we can argue that the probability is at most
$s^2/2^{48}$, where~$s$ is the number of services. But even this bound is
significantly weaker than what we would hope for. In particular, it is much
weaker than the bound in Lemma~1.

Still, this change to the protocol is a viable option for allowing deployed
YubiKeys to be used with multiple servers securely. But what if we're willing to
modify the tokens? In the next section, we propose a way to significantly
improve security. We suspect our proposal will require changing the firmware,
but not the YubiKey hardware.

\textit{No YubiHSM support for this protocol.}
Note that the YubiHSM 1.6 does not support OTP challenge-response.  Recall that
in the Yubico OTP protocol, the expectation is that the payload remain private.
But in the challenge-response protocol, it is necessary that the payload be sent
to the server. In our protocol, the authority can check that the payload matches
what is expected, but this does not appear to be supported.
%
\cpnote{What about YubiHSM 2?}
