\newcommand{\tsprp}{\widetilde{\notionfont{SPRP}}}
\newcommand{\calT}{\mathcal{T}}
\newcommand{\ENCO}{\oraclefont{E}}
\newcommand{\DECO}{\oraclefont{E}^{-1}}
Co-opting the challenge-response mode as described above is a viable way to
use YubiKeys with multiple servers, but there is certainly room for improvement.
%
One way to improve security is to modify the frame data structure so that the
payload is longer. The only essential component is the 3-byte counter, leaving
13 bytes for the payload.
%
However, as long as we're modifying the YubiKey, we can do much better. In the
following, we present a revised OTP protocol, called \emph{OTP2}, that uses a
\emph{tweakable blockcipher} as a building block. Our design may be used in
multiple \emph{modes of operation}, supporting a wide variety of applications
beyond OTP-based authentication.

\newcommand{\otpsec}{\notionfont{IND\$\mbox{-}OTP}}
\newcommand{\OTPO}{\oraclefont{OTP}}
\newcommand{\mode}{\varfont{mode}}
\newcommand{\OTP}{\schemefont{otp}}
\newcommand{\INIT}{\schemefont{init}}
\newcommand{\calM}{\setfont{M}}
\begin{figure}
  \newcommand{\fmtframe}{\schemefont{frame}}
  \twoColsUnbalanced{0.38}{0.58}
  {
    \underline{$\Exp{\tsprp}_{E,b}(\advD)$}\\[2pt]
      $K \getsr \calK$;
      $b' \getsr \advD^{\,\ENCO,\DECO}$\\
      return $b'$
    \\[6pt]
    \underline{$\ENCO(T, X)$}\\[2pt]
      if $b=1$ then return $E_K(T, X)$\\
      if $\pi_T = \bot$ then $\pi_T \getsr \Perm(n)$\\
      return $\pi_T(X)$
    \\[14pt]
    \underline{$\DECO(T, Y)$}\\[2pt]
      if $b=1$ then return $E^{-1}_K(T, Y)$\\
      if $\pi_T = \bot$ then $\pi_T \getsr \Perm(n)$\\
      return $\pi^{-1}_T(Y)$
  }
  {
    \underline{$\Exp{\otpsec}_{E,b}(\advD)$}\\[2pt]
       $K \getsr \calK$; $\ct \gets 0$; $\calE \gets \emptyset$;
       $b' \getsr \advD^{\,\OTPO,\OTPO^{-1}}$\\
       return $b'$
    \\[6pt]
    \underline{$\OTPO(T, \mode, m)$}\\[2pt]
      $\ct \gets \ct + 1$;
      $X \gets \fmtframe(\mode, \ct, m)$
        \comment{Outputs an OTP2 frame.}\\
      if $b=1$ then $Y \gets E_K(T, X)$\\
      else $Y \getsr \bits^n$\\
      $\calE \gets \calE \union \{(T, X, Y)\}$; return $Y$
    \\[6pt]
    \underline{$\OTPO^{-1}(T,Y)$}\\[2pt]
      if $(\exists X)\,(T, X, Y) \in \calE$ then return $X$\\
      if $b=1$ then $X \gets E^{-1}_K(T, Y)$\\
      else $X \getsr \bits^n$\\
      $\calE \gets \calE \union \{(T, X, Y)\}$; return $X$
  }
  \caption{\textbf{Left:} $\tsprp$ security and \textbf{Right:} \otpsec security of
  tweakable blockciphers.}
  \label{fig2}
  \vspace{6pt}
  \hrule
\end{figure}
A \emph{tweakable blockcipher}~\cite{liskov2011tweakable} is a deterministic algorithm
$E:\calK\cross\calT\cross\bits^n\to\bits^n$, where $n$ is a positive integer, $\calT$ is a set,
and $\calK$ is a finite set such that for each $K\in\calK$ and $T\in\calT$,
function $E_K(T,\cdot)$ is a bijection, with $E_K^{-1}(T,\cdot)$ denoting its
inverse.
%
The goal is that each tweak induces a permutation that looks random and
independent from the permutations induced by other tweaks. Security may be
formalized by the game defined in the left-hand panel of Figure~\ref{fig2}.

A number of efficient, secure constructions of tweakable blockciphers are known.
The simplest construction, due to Liskov, Rivest, and
Wagner~\cite{liskov2011tweakable} combines a hash function and a blockcipher.
Let $h : \calT \to \bits^n$ be a function sampled from a family of
$\epsilon$-almost-xor-universal hash functions. This means that for every
$T,T'\in\calT$, where $T\ne T'$, and $\Delta\in\bits^n$, the probability that $h(T)\xor h(T') =
\Delta$ is at most $\epsilon$.
%
(There are several well-known families of hash functions having this property.)
Let $E:\calK\cross\bits^n\to\bits^n$ be a standard blockcipher.  The LRW
tweakable blockcipher is defined by $\widetilde{E}_K(T,X) = E_K(X \xor \Delta)
\xor \Delta$, where $\Delta = h(T)$.
%
XORing $h(T)$ with the input ensures that, with high probability, the inputs of
the blockcipher are different for different tweaks. The result is that
blockciphers with different tweaks look independent to a computationally-bounded
adversary.

\heading{The data frame.}
The protocol will use a tweakable blockcipher with arbitrary-length tweaks.
(Many such constructions are known, cf.~\cite{liskov2011tweakable,landescher2012tweak}.)
%
Namely, let $\calK$ be a finite set, $\calT=\bits^*$, $n\in\N$ be a multiple
of~$8$, $N=n/8$, and $E:\calK\cross\calT\cross\bits^n\to\bits^n$ be a tweakable
blockcipher. The OTP2 frame is much simpler than the OTP frame, having only
three components. It is designed to support multiple ``modes of operation'' used
for protocols beyond one-time-password-based authentication. (We will
discuss these shortly.)
\begin{itemize}
  \item integer $X.\mode$ ($X[0]$) --- Specifies the mode of operation for the
    frame. This is used to signal to the authority which protocol is being
    executed.
  \item integer $X.\ct$ ($\substr(X,1,4)$) --- The 3-byte (24-bit) counter. This
    has the same semantics as $(\tctr,\sctr)$ in the OTP protocol; it is
    incremented each time the token is engaged.
  \item string $X.m$ ($\substr(X,4,)$) --- The payload, comprised of $N-4$
    bytes.
\end{itemize}
The semantics of the payload and tweak are determined by the mode of operation; accordingly,
whether the frame is deemed authentic depends on the mode. However, its freshness is
determined by the $X.\ct$ as before, and the authority must deem a non-fresh
frame to be invalid.
%
The format of the enciphered OTP2 is much the same as before, except that it
depends on a tweak. Namely, it is computed as $\otp = \str(I \cat E_K(T,X))$,
where~$T$ is the tweak and~$I$ is the identity of the token.

\newcommand{\fancykey}{\schemefont{key2}}
\heading{The token.}
%
The OTP2 hardware token will only do one thing: encipher data frames.  The token
takes as input the tweak~$T$, an integer $\mode$, and the payload~$m$.  When
activated, it first checks that $|m| = N-4$ and $\mode$ is a valid mode. It
then increments the internal counter, computes the data frame~$X$, computes $Y
\gets E_K(T,X)$, and outputs $Y$.
%
The token may be used in two ways. The first provides the no-client-side-code
functionality of YubiKeys. Though it is inherently insecure when used for
multiple services, we feel it should be supported. The second mode supports the
full suite of OTP2-based protocols. The token has the following interfaces:
\begin{itemize}
  \item $(\otp, \sigma') \gets \fancykey(\stringer{otp0}, \sigma)$. Outputs a
    fully-formed OTP2 in $\mode=0$.
  \item $(Y, \sigma') \gets \fancykey(\stringer{otp}, T, \mode, m, \sigma)$.
    Outputs an enciphered OTP2 data frame.
  \item $I \gets \fancykey(\stringer{id}, \sigma)$. Outputs the public identity
    of the token.
\end{itemize}
%
Each of these ``calls'' use the same state, so that $(\otp, \sigma') \gets
\fancykey(\stringer{otp}, \sigma)$ is equivalent to running
\[
  (Y, \sigma) \gets \fancykey(\stringer{otp}, \emptystr, 0, \byte(0)^{N-4},  \sigma);
  I \gets \fancykey(\stringer{id}, \sigma)
\]
then computing $\otp \gets \str(I \cat Y)$. (See the OTP0 mode below.)

\textit{Physical considerations.}
An interface is \emph{activated} either by the arrival of some inputs or by the
client physically interacting with it.
%
The \stringer{otp} interface is activated as follows. When a payload is waiting,
the token starts blinking. Once the user \emph{engages} the token by pressing
the button, the payload is processed and an enciphered frame is output. Not every
mode of operation requires engagement.
%
The \stringer{otp0} interface is activated when there is no payload waiting and
when the client engages the token.
%
The \stringer{id} may be activated without engagement.
%
\cpnote{Should the token somehow enable ``cancelation?''}

\subsection{Modes of operation}
We describe four modes of operation for the token.
%
\cpnote{This is the most interesting part from a cryptographic perspective. All
of this looks ``OK'' to me, but each definitely needs a rigorous analysis.}

\heading{OTP0 (\textnormal{$\mode=0$}).}
%
The simple, one-time-password mode. The payload consists of an all-zero byte
string $\byte(0)^{N-4}$ and the tweak is $\emptystr$. Upon receipt, the
authority deciphers the frame, checks that the payload is equal to
$\byte(0)^{N-4}$, and checks that the counter is fresh. This is the
``no-client-side-code'' mode that can be invoked via the token's \stringer{otp0}
interface. \emph{It must be used with only one service}, and must be engaged by
the client.

\heading{OTP (\textnormal{$\mode=1$}).}
%
The \emph{request-bounded, one-time-password} mode. Provides authentication of the
client for a particular request. The request might be to login to a service, or
for authorization to access a resource. The resource is encoded by the tweak.
The payload is specified by the protocol, but it is a static value known to the
client and authority. Upon receipt, the authority decrypts and checks that the
payload matches the string it expects.
%
Requires engagement.

\heading{Integrity (\textnormal{$\mode=2$}).} The \emph{integrity} mode. This
can be used to authenticate (i.e. MAC) a message. The payload is
$\byte(0)^{N-4}$ and the tweak is the message.
%
Requires engagement.

The next two modes are used to implement \emph{authenticated encryption with
associated data}. Such a scheme provides privacy and authenticity of a
message~$M$ and authenticity for associated data~$A$. Normally the syntax
requires an explicit nonce, but our scheme will not require a nonce and take
advantage of the token's stateful counter. It resembles the OCB mode of
operation, but has some significant differences.
%
The string~$M\cat\byte(1)\cat\byte(0)^p$, where $p$ is the smallest, positive
integer such that $|M| + 1 + p \equiv 0 \pmod{N-4}$, is divided into
($N-4$)-byte blocks $(M_1, M_2, \ldots, M_\ell)$.  These are XORed together to
get the checksum~$R$.
%
Each block is processed in order in $\mode=3$ (defined below) to get $(C_1, C_2,
\ldots, C_\ell)$. These blocks are XORed together to get~$S$. Then
$R \xor \substr(S,,N-4)$ is processed in $\mode=4$ to get~$T$.  Finally, the
ciphertext $C_1 \cat C_2 \cat \cdots C_\ell \cat T$ is transmitted to the
authority.
%
The authority decrypts each ciphertext block as it arrives, checking that the
counter value is larger for each block. (The counters need not be contiguous.)
The last block is used to determine if the message is authentic. The authority
must output the message only if it is authentic. The adversary deciphers the
block (using~$A$ as the tweak), and checks that that the payload matches the
checksum of the ciphertext and message blocks.
%
\cpnote{Worth comparing this to https://eprint.iacr.org/2013/835.pdf.}

\heading{Transport (\textnormal{$\mode=3$}).}
%
Processes a message block in transport mode. The payload is the message
block~$M_i$ and the tweak is~$\emptystr$. This mode \emph{does not} require
engagement.

\heading{Transport Finish (\textnormal{$\mode=4$}).}
%
Process the checksum of the message in transport mode. The payload is the
truncated checksum $\substr({(R\xor S)},4,)$ and the tweak is the associated
data~$A$. Requires engagement.
%
\if{0}{
  \cpnote{This is the major departure from OCB. In that mode, $A$ is processed
  in PMAC-fashion and added to the tag, and the tweak is derived from the nonce
  and a counter (starting at 0). We don't need to tweak in this way because we
  stuff a counter into the BC call.}
}\fi

\subsection{Security}
The OTP2 is a building block for four modes of operation: simple authentication,
request-bounded authentication, integrity, and transport. Each mode has a
different security goal, and the hope is that the token is a powerful enough
tool to achieve them.
%
Of course, we need a proof of security for each mode. To that end, we have
formulated a notion of security that models the token's operation on the client's
side and the processing of the output on the authority's side.
%
The \otpsec game (right-hand side of Figure~\ref{fig2}) is associated to a
tweakable blockcipher, a bit~$b$, and an adversary~$\advD$. The adversary is
asked to distinguish the output of the token from a random string given
given access to an oracle for the token's \stringer{otp} interface.  On input of
a tweak, mode, and payload, if $b=1$, then the oracle increments a counter,
constructs and enciphers a frame using the provided tweak, and returns the
output. If $b=0$, it outputs a randomly chosen string. It is also given an
oracle for deciphering frames. If $b=1$, then it outputs the deciphered frame;
otherwise it returns a random string.

Intuitively, security in the \otpsec sense implies that the OTP output of the
token may be treated as a uniform random string. On the other hand, if the
authority is asked to decipher an OTP not output by the token --- or perhaps the
OTP is valid, but the query involves the wrong tweak --- then the deciphered
frame will look random. We conjecture that these properties suffice to prove
security of each of the modes of operation.
%
\cpnote{This might be harder to prove than I think. AFAIK, OTP2 constitutes a
\textbf{new security model}. The OTP2 is being used for different purposes
simultaneously. I haven't seen any other crypto primitive used quite as flexibly
as this.}
%
However, we first need to prove the following:

\heading{Conjecture 1.} \emph{The $\tsprp$ security of tweakable
blockcipher~$E$ implies the \otpsec security of~$E$.}

\noindent
We suspect the proof follows from a similar argument used in Lemma~1.
If so, we should get roughly birthday-bound security.


\subsection{Applications}
\label{sec:apps}
We present a few interesting applications in order to motivate the deployment of
OTP2.

\newcommand{\id}{\varfont{id}}
\newcommand{\pw}{\varfont{pw}}
\heading{Two-factor authentication secure against dictionary attacks.}
%
\emph{OTP mode} can be used to avoid sending a password hash in the clear via the
following protocol. Let $\hash$ be a cryptographic hash function with output
length of $N-4$ bytes. The client~$\client$ has a token~$\fancykey$ managed by
authority~$\auth$ and a username-password pair $(\id, \pw)$. The
server~$\server$ knows the identity~$I$ of~$\fancykey$ and $H = \hash(\id \cat
\pw)$.
%
\begin{enumerate}
  \item $\client$ choses a random, $(N-4)$-byte string~$R$.
    %
    It then computes $(Y, \sigma) \gets \fancykey(\stringer{otp}, 1, R
    \xor H, J, \sigma)$, where~$J$ is the identity of the service~$\server$.
    It then computes $I \gets \fancykey(\stringer{id}, \sigma)$, $\otp \gets
    \str(I \cat Y)$, and sends $(R, \otp, \id)$ to~$\server$.

  \item $\server$ looks up the hash~$H$ associated with~$\id$, then computes $V
    = H\xor R$ and sends $(J, V, \otp)$ to~$\auth$.

  \item $\auth$ decodes $\str(I \cat Y) \gets \otp$, looks up the key~$K$ and
    counter~$\ct$ associated with~$I$, then computes $X \gets E^{-1}_K(J, Y)$.
    %
    It checks if~$X.\ct > \ct$ and $X.m = V$: if so, it lets $\ct \gets X.\ct$
    and sends $(J, \otp, \stringer{ok})$ to~$\server$; otherwise it sends $(J,
    \otp, \stringer{invalid})$ to~$\server$.

  \item If~$\auth$'s response says \stringer{ok}, then accept; otherwise reject.
\end{enumerate}
Each message sent between $\server$ and $\auth$ is encrypted and authenticated
using their shared key. (For example, they may use TLS.) This differs from
the OTP (Section~\ref{sec:otp}) and OTP1 (Sectoin~\ref{sec:otp1}) protocols,
which only require these messages to be authenticated.

This protocol has the same round complexity as OTP and OTP1 and incurs only a
bit of extra overhead.
%
A key advantage of this protocol is that~$H$ is never sent in the clear, which
prevents dictionary attacks on the password by~$\auth$.  Dictionary attacks by a
network adversary are prevented by using a secure channel (i.e. authenticated
encryption) between~$\server$ and~$\auth$.

\heading{Fine-grained resource authorization.}
%
Typically, web services have a \emph{course-grained} model of authorization,
meaning once you've provided credentials and logged in, you needn't provide your
credentials again until you've logged out.
%
There are notable exceptions, however. For example, GitHub requires you to
provide your credentials before deleting a repository. To take another example,
Amazon sometimes requires you to re-enter the last four digits of your credit
card number in order to make a purchase. These exceptions motivate a need for
\emph{finer-grained} authorization for access to extra sensitive resources or
to take certain actions.
%
This is a perfect application for \emph{OTP mode}.\footnote{Imagine it: you're
about to delete a GitHub repo, and the website prompts you to stick in your
token. Then a message pops up: ``Are you sure you want to delete this?'' Click
``yes'': ``Engage your token to confirm.''} The OTP1 protocol might be used for
the same purpose. However, the tiny, 6-byte payload used their means collisions
are quite likely.

\heading{Client authentication extension to TLS.}
%
In a typical TLS handshake, the server is authenticated to the client, but not
the other way around. This leaves open the possibility of a man-in-the-middle.
This looks like a job for \emph{integrity mode!}
%
Suppose the client and server are using ephemeral Diffie-Hellman. The
ClientHello message contains the client's key share $g^a$. The client computes
$(T, \sigma) \gets \fancykey(\stringer{otp}, 2, g^a, \sigma)$, then $I \gets
\fancykey(\stringer{id}, \sigma)$, and sends~$\str(I \cat T)$ along with the
ClientHello.
%
The server forwards $g^a$ and $\str(I \cat T)$ to the authority, who computes $X \gets
E^{-1}_K(g^a, T)$ and checks that $X$ is fresh and $X.m = \byte(0)^{N-4}$. If
so, it updates its state and tells the server to accept.

\heading{Key rotation.}
%
The \emph{transport mode} provides a secure channel between the client and the
authority. One way this can be used is to securely update the token's
configuration. Currently, the only way to update the shared secret is to
generate a fresh secret and physically copy it to the client and authority's
systems. The transport mode could be used to \emph{wrap} a key generated on the
client's system, then transmit the wrapped key to the authority. The same
technique can be used to update the counter, the public identity, and any other
state that needs to be shared between the token and authority.

\subsection{Concrete instantiations of the tweakable blockcipher}

Our requirements for the underlying primitive are an infinite tweak space (i.e.,
$\calT=\bits^*$) and a reasonable block size (e.g., 16 bytes). There are a
number of constructions that would suit our needs, but we need to carefully
consider the constraints of the token's hardware platform. YubiKeys already
support AES-128 and HMAC-SHA1. It should be easy to show, under appropriate
assumptions, that
\[
  \widetilde{E}_{K\cat K'}(T, X) = E_K(X \xor \Delta) \xor \Delta,
  \text{where}\, \Delta = \substr({H_{K'}(T)},,16)
\]
is $\tsprp$ secure, where~$E$ denotes AES-128 and~$H$ denotes HMAC-SHA1. The
assumptions are that~$E$ is an \sprp and $H$ is a PRF. This is essentially the
LRW construction; the $\epsilon$-almost-xor-universality is provided by the PRF
security of~$H$.

Since we're using algorithms already supported by the token, we suspect that
OTP2 can be deployed without modifying the YubiKey hardware; only changes to the
firmware should be required. One potential issue is that the secret key is now
twice as long, since $|K \cat K'| = 32$. (HMAC-SHA1 calls for a
16-byte key.)
%
\cpnote{It depends on how the key is stored, I suppose. If it's impossible store
larger keys, then perhaps we can design a new TBC that uses only a 16-byte key.}
%
It's also worth noting that HMAC-SHA1 should not be considered a secure
PRF, since SHA1 is known to be broken. (We have found a collision!) Therefore,
it may be worth replacing SHA1 with SHA256.

\textit{Precomputing~$\Delta$.}
%
The transport and OTP0 modes use $\emptystr$ as the tweak. The value of
$\Delta_\emptystr = \substr({H_{K'}(\emptystr)},,16)$ might be precomputed and
stored on the token, thus avoiding an HMAC-SHA1 evaluation for these modes.
