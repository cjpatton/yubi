\label{sec:tokens}
This section provides an overview of the functionalities exposed by the hardware
tokens, namely the YubiKey (denoted \key) and the YubiHSM (denoted \hsm).
%
Before we begin, we note that the protocol involves three principals: the
\emph{client} being authenticated, the \emph{server} requesting
authentication of the client, and the \emph{authority} that authenticates the
client to the server. The client possesses~\key and the authority possesses~\hsm.

\heading{YubiKey 4.}
%
There are four varieties of tokens: two are have a large profile designed to be
carried by the client, and two have a small profile designed to remain in the
client's machine. Yubico manufactures a USB-A and USB-C variant for each profile
type.
%
All four varieties support the same suite of protocols.

The token has two (re)programmable slots. A slot is configured to run one of a
variety of protocols, including Yubico OTP, OATH-HOTP, challenge-response
(either OTP- or HMAC-based), or it can be programmed to output a static
password. The protocol is activated by pressing (and holding) on the key. The
first slot is activated by pressing the button and immediately letting go; the
second slot is activated by pressing and holding for a few seconds.
%
All of the values stored on the token, including the shared secret, are
generated by the programmer and written to the device. The programmer can choose
to disable configuration; from that point on, security-critical values cannot be
overwritten.
%
\cpnote{Can a slot configured for, say, OTP-based CR, be used in a different
way, say HMAC-based CR?}

The token presents three interfaces to the operating system: the first two are
\emph{human interface devices} (HIDs) and the third is a \emph{smart card}. The
two HIDs correspond to the two slots and are used for the non-interactive
protocols, in particular Yubico OTP, OATH-HOTP, and static password. The
challenge-response protocols require the client to send a message to the device;
this is handled by the smart card interface.
%
\cptodo{Double check with Dave.}

In this work we are interested in OTP and OTP-based challenge-response.
We write execution of the OTP protocol as $(\otp, \sigma') \gets
\key(\stringer{otp}, \sigma)$, where $\otp$ denotes the OTP, $\sigma$ denotes
the key's internal state, and $\sigma'$ denotes the token's update internal
state. (We emphasize that the client gets the OTP, but not the token's state.)
%
Execution of OTP challenge-response is written $(Y, \sigma') \gets
\key(\stringer{chal}, m, \sigma)$, where $m$ is a 6-byte string called the
\emph{challenge}, and $Y$ is a 16-byte string called the \emph{response}.

\heading{YubiHSM 1.6.}
%
The HSM is designed to provide secure management of shared secrets in protocols
involving YubiKeys, but it also offers a few other cryptographic functionalities
that make it useful for other applications. A properly configured YubiHSM can be
used to improve security. The client needn't trust the authority to keep a
secret; she need only trust that the HSM is properly configured.
%
\cpnote{The configuration of the YubiHSM constitutes an intricate attack
surface, one that warrants further study.}
%
All operations are based on
symmetric cryptography. (However, the newer YubiHSM 2, described below, also
supports a few public-key operations.)
%
The token provides the operating system with two serial interfaces: one for
configuration, and the other for executing protocols.
%
\cptodo{Double check this with Dave.}

The token is configured to operate in one of two modes: \emph{HSM} and
\emph{WSAPI}. The latter facilitates easy deployment of Yubico OTP for a limited
number of keys, while the latter offers more features. Changing the mode of
operation requires resetting the token, and resetting requires physical access.
(It also requires physical access to configure the device.)

\textit{HSM.} This mode offers a number of features, including OTP validation,
management of shared secrets, and a handful of generic operations.
%
Each operation is associated with a \emph{key handle} for a key generated by and
stored within the token. (The HSM holds up to 40 keys.) Also associated to each
key is a set of \emph{permitted operations} using that key. All keys 128-bit
strings generated using the HSM's random number generator.
%
\cpnote{The key store itself is ``encrypted using AES-256'' in non-volatile
memory. Where is does the key come from for encrypting the key store?''}
%
Each key can be configured to perform the following operations:

\begin{itemize}
  \item \textit{Key management.} The HSM can be used to wrap keys for external
    storage. Keys can then be unwrapped within the HSM to be used as a
    \emph{temporary key} (accessed with special handle). AES-CCM is used for key
    (un)wrapping.
    %
    \cpnote{It would be interesting to see how Tom's crypto API
    paper~\cite{SSW16} applies to YubiHSM key wrapping.}

  \item \textit{External Yubico OTP validation.} The HSM stores the key, public and private
    identifiers, and counter state associated with each YubiKey it manages. The
    state is stored externally, using AES-CCM is used for confidentiality and
    integrity.

  \item \textit{Internal Yubico OTP validation.} The HSM can store the state of
    up 1024 YubiKeys in its internal database.

  \item \textit{Message authentication.} A key handle can be used to authenciate
    a message using HMAC-SHA1.

  \item \textit{``Plain'' encryption/decryption.} A key may be used in AES-ECB mode.
    %
    \cpnote{This amounts to a blockcipher oracle. If a key that is used for
    AES-CCM is also used for AES-ECB mode, then ciphertext can easily be
    decrypted using an attack by~\cite{kuennemann2012yubisecure}. The HSM is
    still vulnerable to this attack, since a key may be configured to work in
    both modes. I wonder what other ``bad'' configurations there are.}
\end{itemize}
Another feature of the HSM is \textit{pseudorandom number generation}. Entropy
is gathered from timing difference between input/output events. The PRNG can
also be reseeded using entropy provided by the user. The PRNG is also used to
generate keys stored on the HSM.
%
\cpnote{Dave is interested in chocking the entropy of the PRNG. I'm wondering
what algorithm is used.}

\textit{WSAPI.}
%
The operations above can be used to implement the Yubico OTP protocol as
described in Section~\ref{sec:otp} without exposing any sensitive values to the
authority.
%
The WSAPI mode\footnote{See
\url{https://developers.yubico.com/yubikey-val/Validation_Protocol_V2.0.html}.}
allows the protocol to be implemented conveniently for a limited number of
YubiKeys.
%
This is the ``mainstream'' mode of operation for the HSM, as it supports the
Yubico OTP protocol with a minimal attack surface. A major limitation, however,
is that this mode stores all secrets internally; thus, it is only useful for a
limited deployment.

\heading{YubiHSM 2.}
%
Recently, Yubico released its new version of the YubiHSM, which offers a number
of new features. It supports the common crypto API standards Microsoft CNG and
PKCS\#11, in addition to the native API above.
%
\cpnote{We should check that the API is actually the same.}
%
It also supports new crypto operations, such SHA2, RSA- and EC-based
signing/encryption, and attestation of the HSM's state and configuration.
