% related.tex
%
%
\label{sec:related}

Despite their many pitfalls, passwords have long been understood as the most
practical trade-off between security and usability for authentication.
%
Early adopters recognized the need to store passwords
securely~\cite{morris1979password}, yet no manner of storage is adequate if the
password can easily be guessed. This has lead system admnistrators to enforce
policies for selecting passwords (for example, adding special characters, or
requiring a minimum length) and periodically changing them.  In their landmark
user study, Adams and Sasse~\cite{adams1999users} point out that, paradoxically,
these policies \emph{adversely} affect security, because they drive users to
choose weak passwords, or even to write them down.  In light of this finding,
the usability of any technique to enhance password-based authentication has
become a first-class concern.

A number of approaches to supplement (or supplant) passwords have been proposed,
each having unique barriers to adoption. Bonneau et al.~\cite{bonneau2012quest}
point out that \emph{deployability} of the mechanism is a critical factor.
%
In recent years, two-factor authentication (2FA) has emerged as an approach with
a reasonable trade-off between security, usability, and deployability.
Typically the second factor is realized using something in possession of the
user, such as a cell phone or a dedicated hardware token.
%
The simplest protocol works as follows. The authority associates a unique secret to
each client. When the client requests a log in, the authority uses the secret and
current time to generate a short sequence (6-8) of digits, which it sends to the
client's phone. Both this sequence and the password are needed to log in. Each
``one-time password'' (OTP) is valid only for a short time (typically 30
seconds). While this system is easy to deploy, it suffers from a major security
flaw; the OTP is transmitted via SMS, an insecure channel with a number of known
vulnerabilities~\cite{reaves2016sending}. (For example, vulnerabilities in
SS7~\cite{engel2008locating} have been exploited to intercept OTPs in 2FA
systems used for bank accounts~\cite{schwachstelle}.)

An improvement came with a pair of protocols by OATH, the Initiative for Open
Authentication. The first of these is called TOTP~\cite{rfc6238} and works as
above, except the client's phone (or token) shares a key with the server. This
change affectively removes the SMS attack vector.
%
TOTP enjoys wide adoption; a prominent example is the Google Authenticator app
for Android, iOS and other platforms~\cite{googleauth}. The second,
HOTP~\cite{rfc4226}, is a variant of TOTP that uses a counter instead of a
timestamp.
%
An early hardware token employing a protocol similar to TOTP was RSA's
SecurID~\cite{securid}.
%
Yubico OTP is similar to HOTP, in that it generates one-time passwords using a
stateful counter; however, it has several advantages from a security standpoint.
(For one, Yubico OTPs are much longer, and so harder to forge.)

Each of these protocols, including Yubico OTP, has an important drawback; they
require that the authenticating party keeps a secret. Indeed, in 2011, RSA
SecureID's server was breached, leading to the exposure of a number of tokens'
secrets~\cite{kaminsky2011securid}. This lead Yubico to develop a low-cost,
hardware-security module (the YubiHSM) in order to simplify the deployment of a
secure authentication server.
%
Of courese, public-key cryptography affords the opportunity to avoid this
drawback altogether. The FIDO Allience, in collaboration with a number of
industry leaders, has developed the universal two-factor (U2F)
protocol~\cite{u2f}, which usess a private key stored on a hardware token; the
corresponding public key is stored on the server.
%
YubiKeys support this more sophisticated protocol in addition to Yubico OTP. The
latter has the distinct advantage of being easier to deploy; U2F requires the
web browser to support it, since it involves a couple rounds of communications
between the token and the server. Yubico OTP requires no browser support.

Due to its simplicity, Yubico OTP remains an important industry player. As such,
it has received a respectable amount of attention in academia.
%
The first and only formal treatment of the protocol was provided by K\"unnemann and
Steel~\cite{kuennemann2012yubisecure}, but their result is limited in a
fundamental way. They work in a restricted adversarial model, called the
\emph{Dolev-Yao} model~\cite{herzog2005computational}. Roughly speaking, they
prove that no Dolev-Yao adversary seeing an unbounded number of OTPs can recover
the underlying secret. This is a weaker security property than we generally hope
for. In particular, the stronger \emph{Bellare-Rogaway}
model~\cite{bellare1993entity} directly captures an adversary's ability to forge
the users's credentials.  From this perspective, the exact security of Yubico
OTP remains open.

K\"unneman and Steel also pointed out attacks on the YubiHSM that, with a
particular configuration, allows an attacker on the authority's system to
decrypt sate associated with YubiKeys.
%
YubiKey has also been shown to be susceptible to side-channel attacks. Oswald et
al.~\cite{oswald2013side-channel} use minimally-invasive power analysis to
recover the token's secret key for generating OTPs.

\ignore{
\begin{cool}
\cpnote{Maybe cite
\url{http://ieeexplore.ieee.org/document/4625610/?arnumber=4625610}}
\end{cool}
}
